%
%		LAMA_Implementation_Doc
%
\documentclass[pdftex,a4paper,parskip,listof=totoc,bibliography=totoc,onehalfspacing,12pt]{scrreprt}
\usepackage[ngerman]{babel}
\usepackage[utf8]{inputenc}
\usepackage[T1]{fontenc}
\usepackage{lmodern}
\usepackage{textcomp}
\usepackage{multirow}
%\usepackage{graphicx} wird von pdfpages mitgeladen
\usepackage{pict2e}
\usepackage{multicol}
\usepackage[automark]{scrpage2}
\usepackage{setspace}
\usepackage{color}
\usepackage{natbib}
\usepackage{booktabs}
\usepackage{amsmath}
\usepackage{mathtools}			%Zur Benutzung von \mathrlap und \mathllap
\usepackage{amsfonts}			%Zur Benutzung von \mathbb{R} (Anzeige des Symbols für Reelle Zahlen)
\usepackage{float}
\usepackage{longtable}
\usepackage{pdfpages}
\usepackage{eso-pic}
\usepackage{dirtree}
\usepackage{multirow}
\usepackage{gauss}				% Zur Verwendung von \newcommand*\dashline... (Senkrechter Strich in Arrays)
\usepackage[vlined]{algorithm2e}
\usepackage{enumitem}
	\setlist[itemize]{itemsep=-12pt}
	\setlist[enumerate]{itemsep=-12pt}
\usepackage{subcaption}
\usepackage{lscape}		%Zur Benutzung von \begin{landscape} \end{landscape} Zur Darstellung einzelner Seiten im Querformat


\KOMAoptions{twoside}

\usepackage[left=25mm,right=25mm,bottom=35mm, footskip=20mm]{geometry}

\usepackage{siunitx}		%Darstellung von Einheiten
\sisetup{per-mode=fraction}	%Darstellung von Brüchen


\usepackage{listings}
\usepackage{scrhack}		% Komaoption \KOMAoption{listof}{leveldown} wirkt nun auch auf lstlistoflistings 

\lstdefinestyle{MyMatlab}		% Definition von eigenen Listing-styles
{						% Anwenden mit "\lstinputlisting[style=MyMatlab]{../reflexion/Matlab_Skripte/resample.m}"
	language={Matlab},
	captionpos={t},
	inputencoding={latin1},
	showstringspaces={false},
	frame={tlRB},
	basicstyle=\scriptsize
}

\lstdefinestyle{MyC++}
{
	language={C++},
	tabsize={8},
	captionpos={t},
	frame={tlRB},
	basicstyle=\tiny
}

\lstdefinestyle{MyFortran}
{
	language={[77]Fortran},
	captionpos={t},
	frame={tlRB},
	basicstyle=\tiny
}

\lstdefinestyle{MySh}
{
	language={sh},
	captionpos={t},
	frame={tlRB},
	basicstyle=\tiny
}

\lstdefinestyle{MyGnuplot}
{
	language={Gnuplot},
	captionpos={t},
	frame={tlRB},
	basicstyle=\scriptsize
}

\lstset{literate=%
  {Ö}{{\"O}}1
  {Ä}{{\"A}}1
  {Ü}{{\"U}}1
  {ß}{{\ss}}2
  {ü}{{\"u}}1
  {ä}{{\"a}}1
  {ö}{{\"o}}1
}

\usepackage{rotating}
\usepackage{wrapfig}
\usepackage{tikz}
	\usetikzlibrary{shapes}
	\usetikzlibrary{decorations.pathmorphing}
	\usetikzlibrary{decorations.shapes}
	\usetikzlibrary{shapes.geometric}
	\pgfdeclarelayer{edgelayer}
	\pgfdeclarelayer{nodelayer}
	\pgfsetlayers{edgelayer,nodelayer,main}
	\tikzstyle{none}=[inner sep=0pt]
	\tikzset{decorate with/.style={decorate,decoration={shape backgrounds,shape=#1,shape size=2mm}}}

\definecolor{blue}{rgb}{0.01,0.01,0.95}	%HEX: 0202F2
\definecolor{green}{rgb}{0.01,0.65,0.01}	%HEX: 02A602
\definecolor{red}{rgb}{0.95,0.01,0.01}	%HEX: F20202
\definecolor{brown}{rgb}{0.55,0.27,0.07}	%HEX: 8B4513
\definecolor{violet}{rgb}{0.65,0.05,0.7}	%HEX: A60DB3

\setcounter{tocdepth}{1}	%Legt die Anzeigetiefe des Inhaltsverzeichnisses fest

\renewcommand{\theequation}{\arabic{section}.\arabic{equation}} %Die Nummerierung Formeln wird geändert

\renewcommand{\arraystretch}{1.2} %Legt den Faktor fest, um den der Zeilenabstand innerhalb einer Tabelle oder eines Array gedeht wird
\newcommand*\dashline{\hspace{-0.3em}\rotatebox[origin=c]{90}{\scalebox{-0.3}[1]{$-~~-$}}\hspace{-0.3em}}

%\setlength{\oddsidemargin}{20mm} 
%\setlength{\evensidemargin}{20mm} 

\renewcommand*{\chapterheadstartvskip}{\vspace*{-1\baselineskip}}	%legt Abstand vor chapter fest


\makeatother
\usepackage{hyperref}
\definecolor{LinkColor}{rgb}{0,0,0.75}
\hypersetup{                    %Farben der Links im pdf werden festgelegt
colorlinks=true,
linkcolor=LinkColor,
citecolor=LinkColor,
filecolor=LinkColor,
menucolor=LinkColor,
pagecolor=LinkColor,
urlcolor=LinkColor}

\KOMAoption{listof}{leveldown} %Bewirkt, dass listoffigures, listoftables und lstlistoflistings nicht als chapter behandelt werden. Erschenungsbild im Inhaltsverzeichnis und im Dokument selbst (kein automatischer seitenumbruch am Ende der Verzeichnisse).

\pagestyle{scrheadings} %Legt die Art des Seitenformats für alle folgenden Seiten fest. scrheadings: Um bei zweiseitig formatierten Dokumenten (=Bücher, die unterschiedliche "`linke"' und "`rechte"' Seiten haben) den Seitenkopf und -fuß automatisch an die jeweilige Seite anzupassen, gibt es die folgenden Befehle: \ihead, \chead, \ohead, \ifoot, \cfoot, \ofoot für scrheadings wird "`\usepackage{scrpage2}"' benötigt
\clearscrheadfoot %Löscht alle Kopf- und Fußzeilen
\ohead{\headmark}
\automark[section]{chapter}
\ofoot[\pagemark]{\pagemark}		%[plain-Seiten]{normale Seiten}
\setheadsepline{0.4pt} %erzeugt eine Linie der Stärke 0.4pt zwischen Rumpf und Kopfzeile
\setlength{\headsep}{10mm} %Legt den Abstand zwischen der Kopfzeile und dem Rumpf der Seite fest

\hyphenation{CATIA} %definiert die Stellen zum Trennen von Wörter mit "`-"' Beispiel: \hyphenation{er-go-no-mic}

\title{LAMA_Implementation_Doc}
\author{}

\begin{document}

\numberwithin{equation}{chapter} %legt die Formelnummerierung fest


\thispagestyle{empty} %Kopf und Fußzeile wird für diese Seite ausgeschaltet
\newgeometry{left=25mm,right=25mm,bottom=40mm,top=20mm}
\begin{figure}[h] % Gleiten ist durch Verwendung des H verhindert
\begin{flushright}
\includegraphics[scale=0.15]{./images/wave_logo.png}
\end{flushright}
\end{figure}

\begin{center}
\Large{Theorie der Finite Differenzen Simulation\\ im Matrix-Vektor Formalismus}\\
\vspace{0.5cm}

\end{center}

\vfill
\begin{center}
{\Large{\url{wave-toolbox.org}}}


{\small Dokumentation erstellt am \today}
\end{center}

\newpage 
\thispagestyle{empty}
\begin{center}
{\large
Supported by the German Ministry of Education and Research (BMBF) through the project \textbf{WAVE}, grant 01IH15004A.
}
\end{center}

\cleardoublepage

\pagenumbering{Roman}
\setcounter{page}{1}
\restoregeometry


\newpage

\tableofcontents %Inhaltsverzeichnis wird eingefügt \tableofcontens einfügen vor \addcontentsline, damit der Verweis beim Klicken im Dokument richtig funktioniert
\addcontentsline{toc}{section}{Inhaltsverzeichnis} %Inhaltsverzeichnis wird im Inhaltsverzeichnis angezeigt



\pagenumbering{arabic}
\setcounter{page}{1}

\cleardoublepage
\chapter{Physikalische Herleitung der Wellenausbreitung}

In diesem Kapitel werden die theoretischen Grundlagen für die physikalische Beschreibung der Wellenausbreitung in elastischen Medien gelegt. Dazu zählt die allgemeine Spannungs-Dehnungs-Beziehung von Festkörpern bzw. viskosen Fluiden. Des Weiteren wird die Impulserhaltung am infinitesimalen Volumenelement angewendet.

\section{Allgemeiner Spannungszustand}
\label{kap:AllSp}
Anders als nicht viskose Fluide können viskose Fluide und Festkörper unter Schubspannung stehen und die Normalspannung in den 3 Raumrichtungen sind im Allgemeinen verschieden. Dies ist durch die nun vorhandene inneren Reibung zu erklären.

Der Spannungszustand wird anhand eines infinitesimalen Volumenelements $\partial V = \partial x~ \partial y~ \partial z$ in Abbildung \ref{fig:AllgSpZu} dargestellt.

\begin{figure}
\centering
\begin{tikzpicture}
	\begin{pgfonlayer}{nodelayer}
		\node [style=none] (0) at (0, 0) {};
		\node [style=none, label=-180:$x$] (1) at (-1, -1) {};
		\node [style=none, label=-90:$y$] (2) at (2, 0) {};
		\node [style=none, label=-180:$z$] (3) at (0, 2) {};
		\node [style=none] (4) at (5, 0) {};
		\node [style=none] (5) at (3, -2) {};
		\node [style=none] (6) at (9, 0) {};
		\node [style=none] (7) at (7, -2) {};
		\node [style=none] (8) at (5, 4) {};
		\node [style=none] (9) at (9, 4) {};
		\node [style=none] (10) at (3, 2) {};
		\node [style=none] (11) at (7, 2) {};
		\node [style=none, label=-90:\textcolor{red}{$\tau_{12}$}] (12) at (6, 0) {};
		\node [style=none, label=-180:\textcolor{red}{$\tau_{13}$}] (13) at (5, 1) {};
		\node [style=none, label=180:\textcolor{green}{$\sigma_{11}$}] (14) at (4.5, -0.5) {};
		\node [style=none] (15) at (8, 1) {};
		\node [style=none, label=-180:\textcolor{red}{$\tau_{23}$}] (16) at (8, 2) {};
		\node [style=none, label=0:\textcolor{red}{$\tau_{21}$}] (17) at (7.5, 0.5) {};
		\node [style=none, label=-135:\textcolor{green}{$\sigma_{22}$}] (18) at (9, 1) {};
		\node [style=none] (19) at (6, 3) {};
		\node [style=none, label=-180:\textcolor{red}{$\tau_{31}$}] (20) at (5.5, 2.5) {};
		\node [style=none, label=-90:\textcolor{red}{$\tau_{32}$}] (21) at (7, 3) {};
		\node [style=none, label=-135:\textcolor{green}{$\sigma_{33}$}] (22) at (6, 4) {};
	\end{pgfonlayer}
	\begin{pgfonlayer}{edgelayer}
		\draw [->] (0.center) to (1.center);
		\draw [->] (0.center) to (2.center);
		\draw [->] (0.center) to (3.center);
		\draw [line width = 1.2](8.center) to (10.center);
		\draw [line width = 1.2](10.center) to (5.center);
		\draw [line width = 1.2](5.center) to (7.center);
		\draw [line width = 1.2](7.center) to (6.center);
		\draw [line width = 1.2](6.center) to (9.center);
		\draw [line width = 1.2](8.center) to (9.center);
		\draw [line width = 1.2](10.center) to (11.center);
		\draw [line width = 1.2](11.center) to (9.center);
		\draw [line width = 1.2](11.center) to (7.center);
		\draw [->, color=red] (4.center) to (12.center);
		\draw [->, color=green] (4.center) to (14.center);
		\draw [->, color=red] (4.center) to (13.center);
		\draw [->, color=red] (15.center) to (17.center);
		\draw [->, color=red] (15.center) to (16.center);
		\draw [->, color=green] (15.center) to (18.center);
		\draw [->, color=red] (19.center) to (20.center);
		\draw [->, color=red] (19.center) to (21.center);
		\draw [->, color=green] (19.center) to (22.center);
	\end{pgfonlayer}
\end{tikzpicture}
\caption{Darstellung des allgemeinen Spannungszustandes anhand des Volumenelements $\partial V = \partial x~ \partial y~ \partial z$.}
\label{fig:AllgSpZu}
\end{figure}

Bei der Indexnotation $\sigma_{ij}$ bezeichnet $i$ die Richtung der Flächennormalen, auf welche die Spannung wirkt und $j$ die Richtung der Spannung selbst. $\sigma_{ii}$ sind Normalspannungen mit der Konvention $\sigma_{ii} > 0$: Zug, $\sigma_{ij} < 0$: Druck. $\sigma_{ij}$ mit $i \neq j$ werden auch mit $\tau_{ij}$ bezeichnet und stellen die Schubspannungen dar.

Da der Spannungszustand Körper im statischen Gleichgewicht darstellt, ist für jede Spannung auf den drei verdeckten Seiten je eine entgegen gerichtete Gegenspannung vorhanden.

\begin{minipage}[t]{0.4\textwidth}
\begin{equation}
	\sigma_{ij}=\left(
\begin{array}{ccc}
\sigma_{11} & \tau_{12} & \tau_{13}\\
\tau_{21} & \sigma_{22} & \tau_{23}\\
\tau_{31} & \tau_{32} & \sigma_{33}
\end{array}
\right)
\end{equation}
\end{minipage}
\hfill
\begin{minipage}[t]{0.45\textwidth}
Damit das Volumenelement nicht in Rotation gerät muss also gelten, dass $\sigma_{ij} = \sigma_{ji}$ (Drehimpulserhaltung). Daraus folgt, dass $\sigma_{ij}$ symmetrisch ist und \num{6} unabhängige Spannungen besitzt.
\end{minipage}

Durch das Hookesche Gesetz wird der Zusammenhang zwischen Spannung un Dehnung hergestellt. Für kleine Dehnungen gegenüber der Wellenlänge, was bei der Propagation von seismischen Wellen stets gegeben ist, hängt jede Spannungskomponente näherungsweise linear von den Dehnungen ab.

\begin{minipage}[t]{0.4\textwidth}
\begin{equation}
	\sigma_{ij} = C_{ijkl} \epsilon_{kl}
\end{equation}
\end{minipage}
\hfill
\begin{minipage}[t]{0.45\textwidth}
$\sigma_{ij}$: Spannungstensor\\
$\epsilon_{kl}$: Verzerrungstensor\\
$C_{ijkl}$: Elastizitätstensor
\end{minipage}

Der Elastizitätstensor $C$ ist \num{4}-dimensional und hat im Allgemeinen $3^4 = 81$ Elemente. Aufgrund der Symmetrie von $\sigma$ und $\epsilon$ reduziert sich die Zahl der unabhängigen Komponenten auf 36. Aus energetischen Überlegungen reduziert sich diese weiter auf \num{21} und bei isotropen Medien schließlich auf \num{2}.

Somit lässt sich das generalisierte Hookesche Gesetz für linear elastische, isotrope Medien formulieren:
\begin{align}
	\sigma_{ij} &= \lambda \epsilon_{rr} \delta_{ij} + 2 \mu \epsilon_{ij}\label{eqn:GenHook}\\
	\epsilon_{ij} &= \frac{1}{2} \left( \frac{\partial u_i}{\partial x_j} + \frac{\partial u_j}{\partial x_i} \right)\label{eqn:Dehnij}
\end{align}
Hier bezeichnet $\lambda$ die erste Lamé-Konstante und $\mu$ die zweite Lamé-Konstante (auch Schubmodul genannt). $\delta_{ij}$ ist das Kronecker-Delta. $\epsilon_{ij}$ bezeichnet nun den linearisierten Verzerrungstensor. $u_i$ ist die Verschiebung eines Partikels in Richtung $i$.

\section{Impulserhaltung}
Neben der Spannungs-Dehnungs-Beziehung ist die Impulserhaltung (Kraftdichtebilanz) der Kern der elastischen Wellenausbreitung. Sie entspricht prinzipiell der differentiellen Formulierung des dynamischen Gleichgewichts pro Volumeneinheit. Man betrachte dazu ein infinitesimales Volumen mit der Kantenlänge $\partial x_i$. Auf ihn wirken zwei Arten von Kräften, Volumenkraft und Oberflächenkraft. \citep{landau:97}

Die Volumenkraft, Trägheitskraft und Gravitationskraft, wirkt auf den Körper aufgrund seines massebehafteten Volumens.

Die Trägheitskraft lässt sich mit dem zweiten Newtonschen Axiom in seiner ursprünglichen Formulierung schreiben als:
\begin{equation}
	\frac{\mathrm{d} p}{\mathrm{d} t} = \frac{\mathrm{d} (\partial m \cdot v_i)}{\mathrm{d} t} = \underbrace{\frac{\mathrm{d} (\partial m)}{\mathrm{d} t}}_{\mathrlap{\text{$= 0$ , da $m$ unabhängig von $t$}}} v_i + \partial m \frac{\mathrm{d} v_i}{\mathrm{d} t}
\end{equation}
und als Trägheitskraftdichte mit $v_i = v_i (x_k,t)$:
\begin{equation}
	f_{T,i} = \frac{\partial m}{\partial V} \frac{\mathrm{d} v_i}{\mathrm{d} t} = \rho \frac{\mathrm{d} v_i}{\mathrm{d} t} = \rho \bigg[ \frac{\partial v_i}{\partial t} + \underbrace{v_k \frac{\partial v_i}{\partial x_k}}_{\mathrlap{\substack{\text{$\approx 0$ , da Verschiebungen klein}\\\text{gegenüber Wellenlängen}}}} \bigg] \approx \rho \frac{\partial v_i}{\partial t}\label{eqn:TKD}
\end{equation}
Für elastische Raumwellen kann die Gravitationskraft vernachlässigt werden. Sie spielt lediglich für sehr niederfrequente Eigenschwingungen der Erde eine Rolle.

Die Oberflächenkraft wirkt auf den Körper aufgrund des Spannungszustandes. Die Berechnung erfolgt anhand der Kräftebilanz für alle drei Raumrichtungen.

Zunächst wird die Oberflächenkraft als Integral des Spannungsvektors $\vec{s}$ über die Oberfläche eines Volumenelements $\partial V$ dargestellt.
\begin{equation}
	\vec{F}_O = \int_A \vec{s} ~\mathrm{d}A
\end{equation}

Allgemein lässt sich der Spannungsvektor, der auf einer beliebigen Fläche steht, durch Multiplikation des transponierten Spannungstensors mit der Flächennormalen $\vec{n}$ berechnen. Hier ist zu beachten, dass der Spannungstensor aufgrund der Drehimpulserhaltung symmetrisch ist (s. Kapitel \ref{kap:AllSp}) und somit gilt $\sigma^T = \sigma$
\begin{equation}
	\vec{s} = \sigma^\mathrm{T} \vec{n}
\end{equation}
Mit dem gerichteten Flächenelement $\mathrm{d}\vec{A} = \vec{n} ~\mathrm{d}A$ lässt sich die Oberflächenkraft umschreiben.
\begin{equation}
	\vec{F}_O = \int_A \sigma \vec{n} ~\mathrm{d}A = \int_A \sigma \mathrm{d}\vec{A} 
\end{equation}
Der Integralsatz von Gauß $\int_A \vec{F}~\mathrm{d}\vec{A} = \int_V \operatorname{div} \vec{F} ~\mathrm{d}V$ ermöglicht es das Oberflächenintegral in ein Volumenintegral zu überführen. Dies ist hier vorteilhaft, da so durch das Volumenelement geteilt werden kann, wodurch man die gewünschte Größe einer Oberflächenkraftdichte $\vec{f}_{O}$ erhält.
\begin{align}
	\vec{F}_O &= \int_A \sigma \mathrm{d}\vec{A} = \int_V \operatorname{div} \sigma ~\mathrm{d}V\\
	\vec{f}_O &= \frac{\vec{F}_O}{\mathrm{d}V} = \operatorname{div} \sigma~~~~~~~\vec{f}_{O,i} = \frac{\partial \sigma_{ij}}{\partial x_j}\label{eqn:OFKD}
\end{align}
Schließlich kann das Kraftdichtegleichgewicht $f_{T,i} = f_{O,i}$ aus Gleichung \ref{eqn:TKD} und \ref{eqn:OFKD} zur elastischen Bewegungsgleichung ausformuliert werden.
\begin{equation}
	\rho \frac{\partial v_i}{\partial t} =  \frac{\partial \sigma_{ji}}{\partial x_j}\label{eqn:ElBewGl}
\end{equation}
Die drei Gleichungen \ref{eqn:ElBewGl}, \ref{eqn:GenHook} und \ref{eqn:Dehnij} beschreiben die vollständige Wellenausbreitung in linear elastischen, isotropen Medien.

Ableiten der Materialgesetze nach der Zeit führt zusammen mit Gl. \ref{eqn:ElBewGl} auf die sogenannte Spannungs-Geschwindigkeits Formulierung:
\begin{align}
	\frac{\partial\sigma_{ij}}{\partial t} &= \lambda \frac{\epsilon_{rr}}{\partial t} \delta_{ij} + 2 \mu \frac{\epsilon_{ij}}{\partial t}\label{eqn:GenHookdt}\\
	\frac{\partial\epsilon_{ij}}{\partial t} &= \frac{1}{2} \left( \frac{\partial v_i}{\partial x_j} + \frac{\partial v_j}{\partial x_i} \right)\label{eqn:Dehnijdt}
\end{align}

\cleardoublepage
\chapter{Diskretisierung der Wellengleichung}
Zur Lösung der Wellengleichung mit finiten Differenzen müssen die physikalisch kontinuierliche Raum und Zeit in mathematisch diskretisierte Gitter aufgelöst werden. Für die Koordinaten gilt $x=k \cdot \Delta x$, $y=l \cdot \Delta y$, $z=m \cdot \Delta z$, $t=n \cdot \Delta t$. $\Delta x, \Delta y, \Delta z$ bezeichnen die räumlichen Abstände der Gitterpunkte in der jeweiligen Richtung, $\Delta t$ der zeitliche Abstand. Das Gitter hat eine Gesamtgröße von $NX \times NY \times NZ$. Die Simulationsdauer beträgt $NT \cdot \Delta t$.

\section{Definition des \textit{Standard Staggered Grid} (SSG)}
Das SSG \citep{virieux:86,levander:88} ist eine Methode die physikalischen Größen so auf und zwischen Gitterpunkten zu verteilen, damit sich die partiellen Differentialgleichungen der Wellengleichung jeweils auf einen gemeinsamen Punkt im Gitter beziehen. Die Gestalt des SSG in drei Dimensionen ist in Abbildung \ref{fig:StagGrid} dargestellt. Die zeitliche Dimension ist nicht explizit gezeigt. Die Spannungen $\sigma$ liegen zeitlich auf halben Gitterpunkten, die Partikelgeschwindigkeiten $v$ auf ganzen.
\begin{figure}
\begin{tikzpicture}
	\begin{pgfonlayer}{nodelayer}
		\node [style=none, rectangle, color=green, minimum size=3 mm, fill=green, draw,label=90:{$k,l,m$}] (0) at (0, 0) {};
		\node [style=none, circle, color=orange, minimum size=3 mm, fill=orange, draw,label=90:{$m+\frac{1}{2}$}] (1) at (2, 1) {};
		\node [style=none, diamond, color=brown, minimum size=4 mm, fill=brown, draw,label=90:{$k+\frac{1}{2}$}] (2) at (4, 0) {};
		\node [style=none, regular polygon,regular polygon sides=5, color=black, minimum size=4 mm, fill=white,line width=0.5 mm, draw] (3) at (6, 1) {};
		\node [style=none,label=0:$z(m)$] (4) at (4, 2) {};
		\node [style=none, regular polygon,regular polygon sides=3, color=red, minimum size=4 mm, fill=red, line width=0.5 mm, draw,label=180:{$l+\frac{1}{2}$}] (5) at (0, -4) {};
		\node [style=none, rectangle, color=violet, minimum size=3 mm, line width=0.5 mm, fill=white, draw] (6) at (4, -4) {};
		\node [style=none, regular polygon,regular polygon sides=3, color=blue, minimum size=4 mm, line width=0.5 mm, fill=white, draw] (7) at (2, -3) {};
		\node [style=none] (8) at (6, -3) {};
		\node [style=none,label=-90:$x(k)$] (9) at (8, 0) {};
		\node [style=none,label=0:$y(l)$] (10) at (0, -6) {};
		\node [style=none, rectangle, color=green, minimum size=3 mm, fill=green, draw,label=right:\textcolor{green}{$\sigma_{xx},\sigma_{yy},\sigma_{zz},\lambda,\mu,\rho$}] (11) at (10,2) {};
		\node [style=none, diamond, color=brown, minimum size=4 mm, fill=brown, draw,label=right:\textcolor{brown}{$v_x$}] (12) at (10,1) {};
		\node [style=none, regular polygon,regular polygon sides=3, color=red, minimum size=4 mm, fill=red, line width=0.5 mm, draw,label=right:\textcolor{red}{$v_y$}] (13) at (10,0) {};
		\node [style=none, circle, color=orange, minimum size=3 mm, fill=orange, draw,label=right:\textcolor{orange}{$v_z$}] (14) at (10,-1) {};
		\node [style=none, rectangle, color=violet, minimum size=3 mm, line width=0.5 mm, fill=white, draw,label=right:\textcolor{violet}{$\sigma_{xy}$}] (15) at (10,-2) {};
		\node [style=none, regular polygon,regular polygon sides=3, color=blue, minimum size=4 mm, line width=0.5 mm, fill=white, draw,label=right:\textcolor{blue}{$\sigma_{yz}$}] (16) at (10,-3) {};
		\node [style=none, regular polygon,regular polygon sides=5, color=black, minimum size=4 mm, fill=white,line width=0.5 mm, draw,label=right:\textcolor{black}{$\sigma_{xz}$}] (17) at (10,-4) {};
	\end{pgfonlayer}
	\begin{pgfonlayer}{edgelayer}
		\draw (0.center) to (1.center);
		\draw (2.center) to (3.center);
		\draw (0.center) to (2.center);
		\draw (1.center) to (3.center);
		\draw (0.center) to (5.center);
		\draw (5.center) to (6.center);
		\draw [dashed](5.center) to (7.center);
		\draw (6.center) to (8.center);
		\draw [dashed](7.center) to (8.center);
		\draw (6.center) to (2.center);
		\draw [dashed](7.center) to (1.center);
		\draw (8.center) to (3.center);
		\draw [->] (1.center) to (4.center);
		\draw [->] (2.center) to (9.center);
		\draw [->] (5.center) to (10.center);
	\end{pgfonlayer}
\end{tikzpicture}
\caption{Darstellung des \textit{Standard Staggered Grid} (SSG) mit der Verteilung von Material- und Wellenfeldparametern in 3-D.}
\label{fig:StagGrid}
\end{figure}

Nach dem Lösungsansatz mit finiten Differenzen werden die analytischen Berechnungen der räumlichen und zeitlichen Ableitungen durch ein gewichtetes Mittel (FD-Operatoren) von Werten auf umgebenden Gitterpunkten genähert. Es stehen FD-Operatoren verschiedener Ordnung zur Verfügung, die eine unterschiedliche Anzahl von Gitterpunkten für die Näherung miteinbeziehen. Je weiter sie greifen, desto besser ist die Näherung, jedoch steigt auch der Berechnungsaufwand.

Für eine praktikable Umsetzung des FD-Schemas erfolgt die Lösung der Wellengleichung in einem rekursiven Zeitschrittverfahren. Dazu werden die zeitlichen Ableitungen von Spannungen und Partikelgeschwindigkeiten durch FD-Operatoren 2. Ordnung ersetzt. Hierbei entstehen je zwei zeitlich aufeinander folgende Werte von $\sigma_{ij}$ bzw. $v_i$.
\begin{align}
	\rho \frac{\partial v_i}{\partial t} &= \frac{\partial \sigma_{ji}}{\partial x_j} ~ &\text{mit} \quad \left.\frac{\partial v_i}{\partial t}\right\rvert^{n+\frac{1}{2}} &\approx \frac{v_i^{n+1} - v_i^{n}}{\Delta t}\notag\\
	&\rightarrow v_i^{n+1} \approx v_i^n + \frac{\Delta t}{\rho}  \left.\frac{\partial \sigma_{ji}}{\partial x_j}\right\rvert^{n+\frac{1}{2}}\label{eqn:ElBewGlDis}\\
	\frac{\partial \sigma_{ij}}{\partial t} &= \lambda \frac{\partial \epsilon_{rr}}{\partial t} \delta_{ij} + 2 \mu \frac{\partial \epsilon_{ij}}{\partial t} ~ &\text{mit} \quad \left.\frac{\partial \sigma_{ij}}{\partial t}\right\rvert^n &\approx \frac{\sigma_{ij}^{n+\frac{1}{2}} - \sigma_{ij}^{n-\frac{1}{2}}}{\Delta t}\notag\\
	&\rightarrow \sigma_{ij}^{n+\frac{1}{2}} \approx \sigma_{ij}^{n-\frac{1}{2}} + \Delta t \cdot \lambda \left.\frac{\partial \epsilon_{rr}}{\partial t}\right\rvert^n \delta_{ij} + 2 \cdot \Delta t \cdot \mu \left.\frac{\partial \epsilon_{ij}}{\partial t}\right\rvert^n
\end{align}
An Gleichung \ref{eqn:ElBewGlDis} ist bereits zu erkennen, wie die Definition der Geschwindigkeit auf ganzen und der Spannungen auf halben Gitterpunkten dazu führt, dass sich die Gleichung auf einen gemeinsamen Punkt beziehen.

Eine kompakte Beschreibung des kompletten FD-Schemas lässt sich durch das Verwenden von Platzhaltern der räumlichen Ableitungen für FD-Operatoren beliebiger Ordnung erzielen. Hochgestellt findet sich der Zeitschritt und tiefgestellt der räumliche Mittelpunkt. Für 2. bzw. 4. Ordnung gilt:
\begin{align}
	\left.\frac{\partial\sigma_{xx}}{\partial x}\right\rvert_{k+\frac{1}{2},l,m}^{n+\frac{1}{2}} &\overset{2.O.}{\coloneqq} \frac{\sigma_{xx,k+1,l,m}^{n+\frac{1}{2}} - \sigma_{xx,k,l,m}^{n+\frac{1}{2}}}{\Delta x}\label{eqn:FDO2O}\\
	\left.\frac{\partial\sigma_{xx}}{\partial x}\right\rvert_{k+\frac{1}{2},l,m}^{n+\frac{1}{2}} &\overset{4.O.}{\coloneqq} \frac{\frac{9}{8}\left(\sigma_{xx,k+1,l,m}^{n+\frac{1}{2}} - \sigma_{xx,k,l,m}^{n+\frac{1}{2}}\right) - \frac{1}{24}\left(\sigma_{xx,k+2,l,m}^{n+\frac{1}{2}} - \sigma_{xx,k-1,l,m}^{n+\frac{1}{2}}\right)}{\Delta x}
\end{align}

Damit lässt sich die elastische Wellengleichung unter Verwendung des SSG wie folgt diskretisieren:
\begin{align*}
	\sigma_{xx,k,l,m}^{n+\frac{1}{2}} &= \sigma_{xx,k,l,m}^{n-\frac{1}{2}} + \Delta t \cdot \lambda_{k,l,m} \left( \left.\frac{\partial v_x}{\partial x}\right\rvert_{k,l,m}^n + \left.\frac{\partial v_y}{\partial y}\right\rvert_{k,l,m}^n + \left.\frac{\partial v_z}{\partial z}\right\rvert_{k,l,m}^n \right) + 2 \cdot \Delta t  \cdot\mu_{k,l,m} \left.\frac{\partial v_x}{\partial x}\right\rvert_{k,l,m}^n\\
	\sigma_{yy,k,l,m}^{n+\frac{1}{2}} &= \sigma_{yy,k,l,m}^{n-\frac{1}{2}} + \Delta t \cdot \lambda_{k,l,m} \left( \left.\frac{\partial v_x}{\partial x}\right\rvert_{k,l,m}^n + \left.\frac{\partial v_y}{\partial y}\right\rvert_{k,l,m}^n + \left.\frac{\partial v_z}{\partial z}\right\rvert_{k,l,m}^n \right) + 2 \cdot \Delta t \cdot \mu_{k,l,m} \left.\frac{\partial v_y}{\partial y}\right\rvert_{k,l,m}^n\\
	\sigma_{zz,k,l,m}^{n+\frac{1}{2}} &= \sigma_{zz,k,l,m}^{n-\frac{1}{2}} + \Delta t \cdot \lambda_{k,l,m} \left( \left.\frac{\partial v_x}{\partial x}\right\rvert_{k,l,m}^n + \left.\frac{\partial v_y}{\partial y}\right\rvert_{k,l,m}^n + \left.\frac{\partial v_z}{\partial z}\right\rvert_{k,l,m}^n \right) + 2 \cdot \Delta t \cdot \mu_{k,l,m} \left.\frac{\partial v_z}{\partial z}\right\rvert_{k,l,m}^n\\
	\sigma_{xy,k+\frac{1}{2},l+\frac{1}{2},m}^{n+\frac{1}{2}} &= \sigma_{xy,k+\frac{1}{2},l+\frac{1}{2},m}^{n-\frac{1}{2}} + \Delta t \cdot \mu_{k+\frac{1}{2},l+\frac{1}{2},m} \left( \left.\frac{\partial v_x}{\partial y}\right\rvert_{k+\frac{1}{2},l+\frac{1}{2},m}^n + \left.\frac{\partial v_y}{\partial x}\right\rvert_{k+\frac{1}{2},l+\frac{1}{2},m}^n \right)\\
	\sigma_{xz,k+\frac{1}{2},l,m+\frac{1}{2}}^{n+\frac{1}{2}} &= \sigma_{xz,k+\frac{1}{2},l,m+\frac{1}{2}}^{n-\frac{1}{2}} + \Delta t \cdot \mu_{k+\frac{1}{2},l,m+\frac{1}{2}} \left( \left.\frac{\partial v_x}{\partial z}\right\rvert_{k+\frac{1}{2},l,m+\frac{1}{2}}^n + \left.\frac{\partial v_z}{\partial x}\right\rvert_{k+\frac{1}{2},l,m+\frac{1}{2}}^n \right)\\
	\sigma_{yz,k,l+\frac{1}{2},m+\frac{1}{2}}^{n+\frac{1}{2}} &= \sigma_{yz,k,l+\frac{1}{2},m+\frac{1}{2}}^{n-\frac{1}{2}} + \Delta t \cdot \mu_{k,l+\frac{1}{2},m+\frac{1}{2}} \left( \left.\frac{\partial v_y}{\partial z}\right\rvert_{k,l+\frac{1}{2},m+\frac{1}{2}}^n + \left.\frac{\partial v_z}{\partial y}\right\rvert_{k,l+\frac{1}{2},m+\frac{1}{2}}^n \right)\\
	v_{x,k+\frac{1}{2},l,m}^{n+1} &= v_{x,k+\frac{1}{2},l,m}^n + \frac{\Delta t}{\rho_{k+\frac{1}{2},l,m}}  \left( \left.\frac{\partial\sigma_{xx}}{\partial x}\right\rvert_{k+\frac{1}{2},l,m}^{n+\frac{1}{2}} + \left.\frac{\partial\sigma_{xy}}{\partial y}\right\rvert_{k+\frac{1}{2},l,m}^{n+\frac{1}{2}} + \left.\frac{\partial\sigma_{xz}}{\partial z}\right\rvert_{k+\frac{1}{2},l,m}^{n+\frac{1}{2}} \right)\\
	v_{y,k,l+\frac{1}{2},m}^{n+1} &= v_{y,k,l+\frac{1}{2},m}^n + \frac{\Delta t}{\rho_{k,l+\frac{1}{2},m}}  \left( \left.\frac{\partial\sigma_{yx}}{\partial x}\right\rvert_{k,l+\frac{1}{2},m}^{n+\frac{1}{2}} + \left.\frac{\partial\sigma_{yy}}{\partial y}\right\rvert_{k,l+\frac{1}{2},m}^{n+\frac{1}{2}} + \left.\frac{\partial\sigma_{yz}}{\partial z}\right\rvert_{k,l+\frac{1}{2},m}^{n+\frac{1}{2}} \right)\\
	v_{z,k,l,m+\frac{1}{2}}^{n+1} &= v_{z,k,l,m+\frac{1}{2}}^n + \frac{\Delta t}{\rho_{k,l,m+\frac{1}{2}}}  \left( \left.\frac{\partial\sigma_{zx}}{\partial x}\right\rvert_{k,l,m+\frac{1}{2}}^{n+\frac{1}{2}} + \left.\frac{\partial\sigma_{zy}}{\partial y}\right\rvert_{k,l,m+\frac{1}{2}}^{n+\frac{1}{2}} + \left.\frac{\partial\sigma_{zz}}{\partial z}\right\rvert_{k,l,m+\frac{1}{2}}^{n+\frac{1}{2}} \right)
\end{align*}


Die Verwendung des SSG erfordert hinsichtlich numerischer Stabilität, dass die Dichte $\rho$ auf Zwischengitterpunkten aus den Dichtewerten der umgebenden ganzzahligen Gitterpunkten arithmetisch gemittelt werden. Für den 2. Lamé Parameter $\mu$ hingegen ist eine harmonische Mittelung notwendig.
\begin{align*}
	\rho_{k+\frac{1}{2},l,m} &= \frac{\rho_{k,l,m} + \rho_{k+1,l,m}}{2} \quad \rho_{k,l+\frac{1}{2},m} = \frac{\rho_{k,l,m} + \rho_{k,l+1,m}}{2} \quad \rho_{k,l,m+\frac{1}{2}} = \frac{\rho_{k,l,m} + \rho_{k,l,m+1}}{2}\\
	\mu_{k+\frac{1}{2},l+\frac{1}{2},m} &= \frac{4}{\mu^{-1}_{k,l,m} + \mu^{-1}_{k+1,l,m} + \mu^{-1}_{k,l+1,m} + \mu^{-1}_{k+1,l+1,m}}\\
	\mu_{k+\frac{1}{2},l,m+\frac{1}{2}} &= \frac{4}{\mu^{-1}_{k,l,m} + \mu^{-1}_{k+1,l,m} + \mu^{-1}_{k,l,m+1} + \mu^{-1}_{k+1,l,m+1}}\\
	\mu_{k,l+\frac{1}{2},m+\frac{1}{2}} &= \frac{4}{\mu^{-1}_{k,l,m} + \mu^{-1}_{k,l+1,m} + \mu^{-1}_{k,l,m+1} + \mu^{-1}_{k,l+1,m+1}}
\end{align*}

\section{Ablauf der Wellenfeldberechnung}
Zur Simulation des Wellenfeldes in Raum und Zeit muss jede der 9 diskretisierten Gleichungen an jedem Ort und zu jeder Zeit ausgewertet werden. Da jeweils der zeitlich vorangegangene Wert benötigt wird muss $n$ stetig erhöht werden. Dagegen spielt die Reihenfolge der räumlichen Abarbeitung keine Rolle. Dagegen werden die Spannungen $\sigma_{ij}^{n+\frac{1}{2}}$ für die Berechnung der Partikelgeschwindigkeiten $v_i^{n+\frac{1}{2}}$ benötigt und müssen daher zuerst berechnet werden.

\begin{algorithm}
\caption{Finite Differenzen Simulationsablauf}
\label{alg:FD}
\For{n=1 \KwTo NT}{
	\For{k=1 \KwTo NX}{
		\For{l=1 \KwTo NY}{
			\For{m=1 \KwTo NZ}{
				Update stresses $\sigma_{ij}$
			}
		}
	}
	\For{k=1 \KwTo NX}{
		\For{l=1 \KwTo NY}{
			\For{m=1 \KwTo NZ}{
				Update velocities $v_i$
			}
		}
	}		
}
\end{algorithm}

\cleardoublepage
\chapter{Matrix-Vektor Formalismus des FD-Schemas}
In diesem Kapitel werden zunächst die Vorteile des neuen Formalismus angesprochen. Anschließend wird die Vektorisierung der Arrays vorgenommen. Ein zentraler Schritt stellt die Auffindung von speziellen Matrizen dar, welche die FD-Operatoren auf $v_i$ und $\sigma_{ij}$ anwenden. Diese werden für ein übersichtliches Beispiel aufgeführt.
\section{Motivation}
Der Matrix-Vektor Formalismus bedeutet zunächst lediglich eine kompakte Schreibweise der identischen Mathematik. Neben einer besseren Übersichtlichkeit ergeben sich die großen Vorteile erst auf Seiten der programmtechnischen Verarbeitung. Es handelt sich um eine universelle Schreibweise in die eine Vielzahl von unterschiedlichen physikalischen Problemen gebracht werden kann. So bietet das Framework \textit{LAMA} \citep{lama:16} eine Softwarestruktur für \textit{C++}, die das Schreiben von hardwareunabhängigen Programmen ermöglicht, welche zudem auf heterogenen Rechnerarchitekturen ausführbar sind. Ein Zugang zu dieser Funktionalität ist das Formulieren von Problemen im Matrix-Vektor Formalismus.
\begin{equation}
	\vec{y} = A~\vec{x}
\end{equation}

\section{Vektorisierung der Arrays}
Wie bereits erwähnt ist die Reihenfolge der räumlichen Abarbeitung irrelevant. Dies ist Voraussetzung damit sich die drei räumlichen Schleifen sowohl für die Berechnung von $v_i$ also auch von $\sigma_{ij}$ durch Matrixoperationen ersetzen lassen. Für das Verständnis diesen Schrittes werden die diskretisierten Gleichungen mit FD-Operatoren 2. Ordnung (vgl. Gleichung \ref{eqn:FDO2O}) ersetzt, hier für eine äquidistante Gittergröße ($\Delta x = \Delta y = \Delta z = \Delta h$). Da in Programmiersprachen keine halben Indizes angesprochen werden können, sind intern all diese auf vorangegangene ganze Indizes verschoben. Für das Update von $v_x$ und $\sigma_{xy}$ gilt damit:
\begin{multline}
	v_{x,k,l,m}^{n+1} = v_{x,k,l,m}^n + \frac{\Delta t}{\rho_{k,l,m} \cdot \Delta h} \cdot\\  \left( \sigma_{xx,k+1,l,m}^{n} - \sigma_{xx,k,l,m}^{n} + \sigma_{xy,k,l,m}^{n} - \sigma_{xy,k,l-1,m}^{n} + \sigma_{xz,k,l,m}^{n} - \sigma_{xz,k,l,m-1}^{n} \right)\label{eqn:exMV1}
\end{multline}

Es ist nun zu erkennen, dass alle physikalischen Größen, die in dreidimensionalen Arrays vorliegen (hier $v_x$, $\sigma_{ij}$ und $\rho$) in Vektoren umgewandelt werden müssen. Das Vorgehen ist Konvention, muss aber konsequent eingehalten werden. Abbildung \ref{fig:ArrayToVec} illustriert die Umwandlung eines $4 \times 3 \times 2$-Arrays in einen Vektor. Zuerst wird die x-Richtung, danach die y-Richtung und zuletzt die z-Richtung projiziert.

\begin{figure}
\centering
\begin{tikzpicture}
	\begin{pgfonlayer}{nodelayer}
		\node [style=none] (0) at (0, 0) {};
		\node [style=none] (1) at (1, -1) {1};
		\node [style=none] (2) at (2, -1) {2};
		\node [style=none] (3) at (3, -1) {3};
		\node [style=none] (4) at (4, -1) {4};
		\node [style=none] (5) at (1, -2) {5};
		\node [style=none] (6) at (2, -2) {6};
		\node [style=none] (7) at (3, -2) {7};
		\node [style=none] (8) at (4, -2) {8};
		\node [style=none] (9) at (1, -3) {9};
		\node [style=none] (10) at (2, -3) {10};
		\node [style=none] (11) at (3, -3) {11};
		\node [style=none] (12) at (4, -3) {12};
		\node [style=none] (13) at (1.5, -0.5) {\scriptsize{13}};
		\node [style=none] (14) at (2.5, -0.5) {\scriptsize{14}};
		\node [style=none] (15) at (3.5, -0.5) {\scriptsize{15}};
		\node [style=none] (16) at (4.5, -0.5) {\scriptsize{16}};
		\node [style=none] (17) at (1.5, -1.5) {\scriptsize{17}};
		\node [style=none] (18) at (2.5, -1.5) {\scriptsize{18}};
		\node [style=none] (19) at (3.5, -1.5) {\scriptsize{19}};
		\node [style=none] (20) at (4.5, -1.5) {\scriptsize{20}};
		\node [style=none] (21) at (1.5, -2.5) {\scriptsize{21}};
		\node [style=none] (22) at (2.5, -2.5) {\scriptsize{22}};
		\node [style=none] (23) at (3.5, -2.5) {\scriptsize{23}};
		\node [style=none] (24) at (4.5, -2.5) {\scriptsize{24}};
		\node [style=none,label=-90:$x(k)$] (25) at (6, 0) {};
		\node [style=none,label=0:$y(l)$] (26) at (0, -4) {};
		\node [style=none,label=0:$z(m)$] (27) at (1, 1) {};
	\end{pgfonlayer}
	\begin{pgfonlayer}{edgelayer}
		\draw [->](0.center) to (25.center);
		\draw [->](0.center) to (26.center);
		\draw [->](0.center) to (27.center);
	\end{pgfonlayer}
\end{tikzpicture}
\caption{Umwandlung eines $4 \times 3 \times 2$-Arrays in einen Vektor. Dargestellt ist der Index des sich ergebenden Vektors.}
\label{fig:ArrayToVec}
\end{figure}

\section{Einführung der FD-Koeffizienten Matrizen}
Der größter Schritt des neuen Formalismus ist das Auffinden der FD-Koeffizienten Matrizen $\underline{D}$.

\subsection{Auswirkung der Indexverschiebung}
Aufgrund der Verschiebung aller halben Indizes um $-\frac{1}{2}$ muss bei der Anwendung unterschieden werden. Liegt das ursprüngliche Zentrum eines FD-Operators in seiner Ableitungsrichtung auf einem ganzen Gitterpunkt, wie z.B. bei dem Update von $\sigma_{xx,k,l,m}$ wo das Zentrum von $\left. \frac{\partial v_x}{\partial x} \right\rvert_{k,l,m}$ in der Ableitungsrichtung $x$ auf $k$ liegt, so greift der FD-Operator 2. Ordnung theoretisch auf den Term $v_{x,k+\frac{1}{2},l,m} - v_{x,k-\frac{1}{2},l,m}$ zurück. Nach der Reduzierung der halben Indizes um $-\frac{1}{2}$ verschiebt sich die Mitte des Operators ebenfalls nach links und das Ergebnis von $\sigma_{xx,k,l,m}$ an der Stelle $k$ wird aus $v_x$-Werten an den Stellen $k-1$ und $k$ berechnet (s. Abbildung \ref{fig:IndShift} links). Diese Geometrie ist bei allen $\frac{\partial v_i}{\partial x_i}$ sowie $\frac{\partial \sigma_{ij}}{\partial x_j}$ mit $i\neq j$ gegeben. Die hierzu passende FD-Koeffizienten Matrix wird mit $\underline{D}_{\,i,\mathrm{b}}$ mit b für \textit{backward} bezeichnet. Das $i$ steht für die Ableitungsrichtung.

Der hiervon zu unterscheidende Fall liegt vor, wenn das Zentrum eines FD-Operators in seiner Ableitungsrichtung auf einem halben Gitterpunkt, wie z.B. bei dem Update von $\sigma_{xy,k+\frac{1}{2},l+\frac{1}{2},m}$ wo das Zentrum von $\left. \frac{\partial v_x}{\partial y} \right\rvert_{k+\frac{1}{2},l+\frac{1}{2},m}$ in der Ableitungsrichtung $y$ auf $l+\frac{1}{2}$ liegt. Dann greift der FD-Operator 2. Ordnung theoretisch auf den Term $v_{x,k+\frac{1}{2},l+1,m} - v_{x,k+\frac{1}{2},l,m}$ zurück. Diese $v_x$ Werte müssen in $y$-Richtung nicht verschoben werden. Jedoch wird der hiermit berechnete Wert $\sigma_{xy,k+\frac{1}{2},l+\frac{1}{2},m}$ auf $\sigma_{xy,k,l,m}$, also in $y$-Richtung, verschoben, womit das Zentrum des FD-Operators nun aus Sicht von $\sigma_{xy,k,l,m}$ rechts befindet (s. Abbildung \ref{fig:IndShift} rechts). Diese Verschiebung findet sich bei allen $\frac{\partial v_i}{\partial x_j}$ mit $i\neq j$ und $\frac{\partial \sigma_{ii}}{\partial x_i}$. Hierfür lautet die Bezeichnung $\underline{D}_{\,i,\mathrm{f}}$ mit f für \textit{forward}.

\begin{figure}
\begin{tikzpicture}
	\begin{pgfonlayer}{nodelayer}
		\node [style=none, circle, color=green, minimum size=3 mm, fill=green, draw] (0) at (0, 0) {};
		\node [style=none] (1) at (2, 0) {};
		\node [style=none, circle, color=green, minimum size=3 mm, fill=green, draw] (2) at (4, 0) {};
		\node [style=none, circle, color=orange, minimum size=3 mm, fill=white, draw,label=90:{$k-\frac{1}{2}$}] (3) at (2, 2) {};
		\node [style=none, circle, color=orange, minimum size=3 mm, fill=orange, draw,label=90:{$k$}] (4) at (4, 2) {};
		\node [style=none, circle, color=orange, minimum size=3 mm, fill=white, draw,label=90:{$k+\frac{1}{2}$}] (5) at (6, 2) {};
		\node [style=none, circle, color=orange, minimum size=3 mm, fill=orange, draw,label=90:{$k$}] (6) at (10, 2) {};
		\node [style=none, circle, color=orange, minimum size=3 mm, fill=white, draw,label=90:{$k+\frac{1}{2}$}] (7) at (12, 2) {};
		\node [style=none, circle, color=orange, minimum size=3 mm, fill=orange, draw,label=90:{$k+1$}] (8) at (14, 2) {};
		\node [style=none, circle, color=green, minimum size=3 mm, fill=green, draw] (9) at (10, 0) {};
		\node [style=none, circle, color=green, minimum size=3 mm, fill=green, draw] (10) at (14, 0) {};
		\node [style=none] (11) at (1.5, 1.5) {};
		\node [style=none] (12) at (0.5, 0.5) {};
		\node [style=none] (13) at (5.5, 1.5) {};
		\node [style=none] (14) at (4.5, 0.5) {};
		\node [style=none] (15) at (11.5, 1.5) {};
		\node [style=none] (16) at (10.5, 0.5) {};
	\end{pgfonlayer}
	\begin{pgfonlayer}{edgelayer}
		\draw [->](11.center) to (12.center);
		\draw [->](13.center) to (14.center);
		\draw [->](15.center) to (16.center);
	\end{pgfonlayer}
\end{tikzpicture}
\caption{Wirkung der Indexverschiebung auf die Lage der FD-Operatoren. Links: Ursprünglicher Mittelpunkt liegt auf $k$ und verschiebt sich relativ nach rechts, wodurch er zurückgreift ($\underline{D}_{\,i,b}$). Rechts: Ursprünglicher Mittelpunkt liegt auf $k+\frac{1}{2}$ und verschiebt sich relativ nach links, wodurch er vorgreift ($\underline{D}_{\,i,f}$).}
\label{fig:IndShift}
\end{figure}


Die Auswirkung dieser Unterscheidung äußert sich in einer Verschiebung der FD-Koeffizienten in der Matrix, was sich allgemein durch $\underline{D}_{\,i,f} = -\underline{D}_{\,i,b}^T$ ausdrücken lässt.

\subsection{Vervollständigung des Formalismus}

Ein Beispiel für eine Updategleichung im neuen Matrix-Vektor Formalismus basiert auf Gleichung \ref{eqn:exMV1}, wo die Indexverschiebung bereits durchgeführt wurde:
\begin{equation}
	\vec{v}_x^{\,n+1} = \vec{v}_x^{\,n} + \frac{\Delta t}{\Delta h} \cdot \mathrm{diag} \left( \vec{\rho}_\mathrm{inv}^{\,T} \right) \cdot \left( \underline{D}_{\,x,f}^2 \vec{\sigma}_{xx}^{\,n} + \underline{D}_{\,y,b}^2 \vec{\sigma}_{xy}^{\,n} + \underline{D}_{\,z,b}^2 \vec{\sigma}_{xz}^{\,n} \right)\label{eqn:exMV2}
\end{equation}
Die Diagonalmatrix $\mathrm{diag} \left( \vec{\rho}_\mathrm{inv}^{\,T} \right) = \mathrm{diag} \left( \rho_1^{-1}, \rho_2^{-1}, \rho_3^{-1}, \dots , \rho_{NX \cdot NY \cdot NZ}^{-1} \right)$ enthält die Werte des Dichtemodells. $\underline{D}_{\,i,b}^2$ bzw. $\underline{D}_{\,i,f}^2$  beinhaltet die FD-Koeffizienten 2. Ordnung für die Ableitungsrichtung $i$. Die Aufgabe besteht nun darin, die Matrizen $\underline{D}_{\,i}^2$ zu finden, sodass sie die Modellvektoren $\vec{\sigma}_{ij}$ auf die FD-Operatoren abbilden. 

Nachfolgend werden die Matrizen $\underline{D}_{\,i,f}^2$ und $\underline{D}_{\,i,f}^4$ dargestellt um sich die Auswirkungen der FD-Ordnung machen zu können. Man führe die Multiplikation folgender Matrizen in Gleichung \ref{eqn:exMV2} für obiges Beispiel aus und überzeuge sich vom Ergebnis. Anschließend folgt die Entwicklung der $\underline{D}_{\,i,b}$.
\begin{align*}
	\underline{D}_{\,x,f}^2 &= \begin{psmallmatrix}
\text{-}1 & 1 & 0 & 0 &\dashline& 0 & 0 & 0 & 0 && 0 & 0 & 0 & 0 && 0 & 0 & 0 & 0 && 0 & 0 & 0 & 0 && 0 & 0 & 0 & 0 \\
0 & \text{-}1 & 1 & 0 &\dashline& 0 & 0 & 0 & 0 && 0 & 0 & 0 & 0 && 0 & 0 & 0 & 0 && 0 & 0 & 0 & 0 && 0 & 0 & 0 & 0 \\
0 & 0 & \text{-}1 & 1 &\dashline& 0 & 0 & 0 & 0 && 0 & 0 & 0 & 0 && 0 & 0 & 0 & 0 && 0 & 0 & 0 & 0 && 0 & 0 & 0 & 0 \\
0 & 0 & 0 & \text{-}1 &\dashline& 0 & 0 & 0 & 0 && 0 & 0 & 0 & 0 && 0 & 0 & 0 & 0 && 0 & 0 & 0 & 0 && 0 & 0 & 0 & 0 \\\hline
0 & 0 & 0 & 0 &\dashline& \text{-}1 & 1 & 0 & 0 &\dashline& 0 & 0 & 0 & 0 && 0 & 0 & 0 & 0 && 0 & 0 & 0 & 0 && 0 & 0 & 0 & 0 \\
0 & 0 & 0 & 0 &\dashline& 0 & \text{-}1 & 1 & 0 &\dashline& 0 & 0 & 0 & 0 && 0 & 0 & 0 & 0 && 0 & 0 & 0 & 0 && 0 & 0 & 0 & 0 \\
0 & 0 & 0 & 0 &\dashline& 0 & 0 & \text{-}1 & 1 &\dashline& 0 & 0 & 0 & 0 && 0 & 0 & 0 & 0 && 0 & 0 & 0 & 0 && 0 & 0 & 0 & 0 \\
0 & 0 & 0 & 0 &\dashline& 0 & 0 & 0 & \text{-}1 &\dashline& 0 & 0 & 0 & 0 && 0 & 0 & 0 & 0 && 0 & 0 & 0 & 0 && 0 & 0 & 0 & 0 \\\hline
0 & 0 & 0 & 0 && 0 & 0 & 0 & 0 &\dashline& \text{-}1 & 1 & 0 & 0 &\dashline& 0 & 0 & 0 & 0 && 0 & 0 & 0 & 0 && 0 & 0 & 0 & 0 \\
0 & 0 & 0 & 0 && 0 & 0 & 0 & 0 &\dashline& 0 & \text{-}1 & 1 & 0 &\dashline& 0 & 0 & 0 & 0 && 0 & 0 & 0 & 0 && 0 & 0 & 0 & 0 \\
0 & 0 & 0 & 0 && 0 & 0 & 0 & 0 &\dashline& 0 & 0 & \text{-}1 & 1 &\dashline& 0 & 0 & 0 & 0 && 0 & 0 & 0 & 0 && 0 & 0 & 0 & 0 \\
0 & 0 & 0 & 0 && 0 & 0 & 0 & 0 &\dashline& 0 & 0 & 0 & \text{-}1 &\dashline& 0 & 0 & 0 & 0 && 0 & 0 & 0 & 0 && 0 & 0 & 0 & 0 \\\hline
0 & 0 & 0 & 0 && 0 & 0 & 0 & 0 && 0 & 0 & 0 & 0 &\dashline& \text{-}1 & 1 & 0 & 0 &\dashline& 0 & 0 & 0 & 0 && 0 & 0 & 0 & 0 \\
0 & 0 & 0 & 0 && 0 & 0 & 0 & 0 && 0 & 0 & 0 & 0 &\dashline& 0 & \text{-}1 & 1 & 0 &\dashline& 0 & 0 & 0 & 0 && 0 & 0 & 0 & 0 \\
0 & 0 & 0 & 0 && 0 & 0 & 0 & 0 && 0 & 0 & 0 & 0 &\dashline& 0 & 0 & \text{-}1 & 1 &\dashline& 0 & 0 & 0 & 0 && 0 & 0 & 0 & 0 \\
0 & 0 & 0 & 0 && 0 & 0 & 0 & 0 && 0 & 0 & 0 & 0 &\dashline& 0 & 0 & 0 & \text{-}1 &\dashline& 0 & 0 & 0 & 0 && 0 & 0 & 0 & 0 \\\hline
0 & 0 & 0 & 0 && 0 & 0 & 0 & 0 && 0 & 0 & 0 & 0 && 0 & 0 & 0 & 0 &\dashline& \text{-}1 & 1 & 0 & 0 &\dashline& 0 & 0 & 0 & 0 \\
0 & 0 & 0 & 0 && 0 & 0 & 0 & 0 && 0 & 0 & 0 & 0 && 0 & 0 & 0 & 0 &\dashline& 0 & \text{-}1 & 1 & 0 &\dashline& 0 & 0 & 0 & 0 \\
0 & 0 & 0 & 0 && 0 & 0 & 0 & 0 && 0 & 0 & 0 & 0 && 0 & 0 & 0 & 0 &\dashline& 0 & 0 & \text{-}1 & 1 &\dashline& 0 & 0 & 0 & 0 \\
0 & 0 & 0 & 0 && 0 & 0 & 0 & 0 && 0 & 0 & 0 & 0 && 0 & 0 & 0 & 0 &\dashline& 0 & 0 & 0 & \text{-}1 &\dashline& 0 & 0 & 0 & 0 \\\hline
0 & 0 & 0 & 0 && 0 & 0 & 0 & 0 && 0 & 0 & 0 & 0 && 0 & 0 & 0 & 0 && 0 & 0 & 0 & 0 &\dashline& \text{-}1 & 1 & 0 & 0 \\
0 & 0 & 0 & 0 && 0 & 0 & 0 & 0 && 0 & 0 & 0 & 0 && 0 & 0 & 0 & 0 && 0 & 0 & 0 & 0 &\dashline& 0 & \text{-}1 & 1 & 0 \\
0 & 0 & 0 & 0 && 0 & 0 & 0 & 0 && 0 & 0 & 0 & 0 && 0 & 0 & 0 & 0 && 0 & 0 & 0 & 0 &\dashline& 0 & 0 & \text{-}1 & 1 \\
0 & 0 & 0 & 0 && 0 & 0 & 0 & 0 && 0 & 0 & 0 & 0 && 0 & 0 & 0 & 0 && 0 & 0 & 0 & 0 &\dashline& 0 & 0 & 0 & \text{-}1 \\
\end{psmallmatrix}
\end{align*}

\begin{align*}
	\underline{D}_{\,x,f}^4 &= \begin{psmallmatrix}
\text{-}\frac{9}{8} & \frac{9}{8} & \text{-}\frac{1}{24} & 0 &\dashline& 0 & 0 & 0 & 0 && 0 & 0 & 0 & 0 && 0 & 0 & 0 & 0 && 0 & 0 & 0 & 0 && 0 & 0 & 0 & 0 \\
\frac{1}{24} & \text{-}\frac{9}{8} & \frac{9}{8} & \text{-}\frac{1}{24} &\dashline& 0 & 0 & 0 & 0 && 0 & 0 & 0 & 0 && 0 & 0 & 0 & 0 && 0 & 0 & 0 & 0 && 0 & 0 & 0 & 0 \\
0 & \frac{1}{24} & \text{-}\frac{9}{8} & \frac{9}{8} &\dashline& 0 & 0 & 0 & 0 && 0 & 0 & 0 & 0 && 0 & 0 & 0 & 0 && 0 & 0 & 0 & 0 && 0 & 0 & 0 & 0 \\
0 & 0 & \frac{1}{24} & \text{-}\frac{9}{8} &\dashline& 0 & 0 & 0 & 0 && 0 & 0 & 0 & 0 && 0 & 0 & 0 & 0 && 0 & 0 & 0 & 0 && 0 & 0 & 0 & 0 \\\hline
0 & 0 & 0 & 0 &\dashline& \text{-}\frac{9}{8} & \frac{9}{8} & \text{-}\frac{1}{24} & 0 &\dashline& 0 & 0 & 0 & 0 && 0 & 0 & 0 & 0 && 0 & 0 & 0 & 0 && 0 & 0 & 0 & 0 \\
0 & 0 & 0 & 0 &\dashline& \frac{1}{24} & \text{-}\frac{9}{8} & \frac{9}{8} & \text{-}\frac{1}{24} &\dashline& 0 & 0 & 0 & 0 && 0 & 0 & 0 & 0 && 0 & 0 & 0 & 0 && 0 & 0 & 0 & 0 \\
0 & 0 & 0 & 0 &\dashline& 0 & \frac{1}{24} & \text{-}\frac{9}{8} & \frac{9}{8} &\dashline& 0 & 0 & 0 & 0 && 0 & 0 & 0 & 0 && 0 & 0 & 0 & 0 && 0 & 0 & 0 & 0 \\
0 & 0 & 0 & 0 &\dashline& 0 & 0 & \frac{1}{24} & \text{-}\frac{9}{8} &\dashline& 0 & 0 & 0 & 0 && 0 & 0 & 0 & 0 && 0 & 0 & 0 & 0 && 0 & 0 & 0 & 0 \\\hline
0 & 0 & 0 & 0 && 0 & 0 & 0 & 0 &\dashline& \text{-}\frac{9}{8} & \frac{9}{8} & \text{-}\frac{1}{24} & 0 &\dashline& 0 & 0 & 0 & 0 && 0 & 0 & 0 & 0 && 0 & 0 & 0 & 0 \\
0 & 0 & 0 & 0 && 0 & 0 & 0 & 0 &\dashline& \frac{1}{24} & \text{-}\frac{9}{8} & \frac{9}{8} & \text{-}\frac{1}{24} &\dashline& 0 & 0 & 0 & 0 && 0 & 0 & 0 & 0 && 0 & 0 & 0 & 0 \\
0 & 0 & 0 & 0 && 0 & 0 & 0 & 0 &\dashline& 0 & \frac{1}{24} & \text{-}\frac{9}{8} & \frac{9}{8} &\dashline& 0 & 0 & 0 & 0 && 0 & 0 & 0 & 0 && 0 & 0 & 0 & 0 \\
0 & 0 & 0 & 0 && 0 & 0 & 0 & 0 &\dashline& 0 & 0 & \frac{1}{24} & \text{-}\frac{9}{8} &\dashline& 0 & 0 & 0 & 0 && 0 & 0 & 0 & 0 && 0 & 0 & 0 & 0 \\\hline
0 & 0 & 0 & 0 && 0 & 0 & 0 & 0 && 0 & 0 & 0 & 0 &\dashline& \text{-}\frac{9}{8} & \frac{9}{8} & \text{-}\frac{1}{24} & 0 &\dashline& 0 & 0 & 0 & 0 && 0 & 0 & 0 & 0 \\
0 & 0 & 0 & 0 && 0 & 0 & 0 & 0 && 0 & 0 & 0 & 0 &\dashline& \frac{1}{24} & \text{-}\frac{9}{8} & \frac{9}{8} & \text{-}\frac{1}{24} &\dashline& 0 & 0 & 0 & 0 && 0 & 0 & 0 & 0 \\
0 & 0 & 0 & 0 && 0 & 0 & 0 & 0 && 0 & 0 & 0 & 0 &\dashline& 0 & \frac{1}{24} & \text{-}\frac{9}{8} & \frac{9}{8} &\dashline& 0 & 0 & 0 & 0 && 0 & 0 & 0 & 0 \\
0 & 0 & 0 & 0 && 0 & 0 & 0 & 0 && 0 & 0 & 0 & 0 &\dashline& 0 & 0 & \frac{1}{24} & \text{-}\frac{9}{8} &\dashline& 0 & 0 & 0 & 0 && 0 & 0 & 0 & 0 \\\hline
0 & 0 & 0 & 0 && 0 & 0 & 0 & 0 && 0 & 0 & 0 & 0 && 0 & 0 & 0 & 0 &\dashline& \text{-}\frac{9}{8} & \frac{9}{8} & \text{-}\frac{1}{24} & 0 &\dashline& 0 & 0 & 0 & 0 \\
0 & 0 & 0 & 0 && 0 & 0 & 0 & 0 && 0 & 0 & 0 & 0 && 0 & 0 & 0 & 0 &\dashline& \frac{1}{24} & \text{-}\frac{9}{8} & \frac{9}{8} & \text{-}\frac{1}{24} &\dashline& 0 & 0 & 0 & 0 \\
0 & 0 & 0 & 0 && 0 & 0 & 0 & 0 && 0 & 0 & 0 & 0 && 0 & 0 & 0 & 0 &\dashline& 0 & \frac{1}{24} & \text{-}\frac{9}{8} & \frac{9}{8} &\dashline& 0 & 0 & 0 & 0 \\
0 & 0 & 0 & 0 && 0 & 0 & 0 & 0 && 0 & 0 & 0 & 0 && 0 & 0 & 0 & 0 &\dashline& 0 & 0 & \frac{1}{24} & \text{-}\frac{9}{8} &\dashline& 0 & 0 & 0 & 0 \\\hline
0 & 0 & 0 & 0 && 0 & 0 & 0 & 0 && 0 & 0 & 0 & 0 && 0 & 0 & 0 & 0 && 0 & 0 & 0 & 0 &\dashline& \text{-}\frac{9}{8} & \frac{9}{8} & \text{-}\frac{1}{24} & 0 \\
0 & 0 & 0 & 0 && 0 & 0 & 0 & 0 && 0 & 0 & 0 & 0 && 0 & 0 & 0 & 0 && 0 & 0 & 0 & 0 &\dashline& \frac{1}{24} & \text{-}\frac{9}{8} & \frac{9}{8} & \text{-}\frac{1}{24} \\
0 & 0 & 0 & 0 && 0 & 0 & 0 & 0 && 0 & 0 & 0 & 0 && 0 & 0 & 0 & 0 && 0 & 0 & 0 & 0 &\dashline& 0 & \frac{1}{24} & \text{-}\frac{9}{8} & \frac{9}{8} \\
0 & 0 & 0 & 0 && 0 & 0 & 0 & 0 && 0 & 0 & 0 & 0 && 0 & 0 & 0 & 0 && 0 & 0 & 0 & 0 &\dashline& 0 & 0 & \frac{1}{24} & \text{-}\frac{9}{8} \\
\end{psmallmatrix}
\end{align*}

\begin{align*}
	\underline{D}_{\,y,f}^2 &= \begin{psmallmatrix}
\text{-}1 & 0 & 0 & 0 & 1 & 0 & 0 & 0 & 0 & 0 & 0 & 0 &\dashline& 0 & 0 & 0 & 0 & 0 & 0 & 0 & 0 & 0 & 0 & 0 & 0 \\
0 & \text{-}1 & 0 & 0 & 0 & 1 & 0 & 0 & 0 & 0 & 0 & 0 &\dashline& 0 & 0 & 0 & 0 & 0 & 0 & 0 & 0 & 0 & 0 & 0 & 0 \\
0 & 0 & \text{-}1 & 0 & 0 & 0 & 1 & 0 & 0 & 0 & 0 & 0 &\dashline& 0 & 0 & 0 & 0 & 0 & 0 & 0 & 0 & 0 & 0 & 0 & 0 \\
0 & 0 & 0 & \text{-}1 & 0 & 0 & 0 & 1 & 0 & 0 & 0 & 0 &\dashline& 0 & 0 & 0 & 0 & 0 & 0 & 0 & 0 & 0 & 0 & 0 & 0 \\
0 & 0 & 0 & 0 & \text{-}1 & 0 & 0 & 0 & 1 & 0 & 0 & 0 &\dashline& 0 & 0 & 0 & 0 & 0 & 0 & 0 & 0 & 0 & 0 & 0 & 0 \\
0 & 0 & 0 & 0 & 0 & \text{-}1 & 0 & 0 & 0 & 1 & 0 & 0 &\dashline& 0 & 0 & 0 & 0 & 0 & 0 & 0 & 0 & 0 & 0 & 0 & 0 \\
0 & 0 & 0 & 0 & 0 & 0 & \text{-}1 & 0 & 0 & 0 & 1 & 0 &\dashline& 0 & 0 & 0 & 0 & 0 & 0 & 0 & 0 & 0 & 0 & 0 & 0 \\
0 & 0 & 0 & 0 & 0 & 0 & 0 & \text{-}1 & 0 & 0 & 0 & 1 &\dashline& 0 & 0 & 0 & 0 & 0 & 0 & 0 & 0 & 0 & 0 & 0 & 0 \\
0 & 0 & 0 & 0 & 0 & 0 & 0 & 0 & \text{-}1 & 0 & 0 & 0 &\dashline& 0 & 0 & 0 & 0 & 0 & 0 & 0 & 0 & 0 & 0 & 0 & 0 \\
0 & 0 & 0 & 0 & 0 & 0 & 0 & 0 & 0 & \text{-}1 & 0 & 0 &\dashline& 0 & 0 & 0 & 0 & 0 & 0 & 0 & 0 & 0 & 0 & 0 & 0 \\
0 & 0 & 0 & 0 & 0 & 0 & 0 & 0 & 0 & 0 & \text{-}1 & 0 &\dashline& 0 & 0 & 0 & 0 & 0 & 0 & 0 & 0 & 0 & 0 & 0 & 0 \\
0 & 0 & 0 & 0 & 0 & 0 & 0 & 0 & 0 & 0 & 0 & \text{-}1 &\dashline& 0 & 0 & 0 & 0 & 0 & 0 & 0 & 0 & 0 & 0 & 0 & 0 \\\hline
0 & 0 & 0 & 0 & 0 & 0 & 0 & 0 & 0 & 0 & 0 & 0 &\dashline& \text{-}1 & 0 & 0 & 0 & 1 & 0 & 0 & 0 & 0 & 0 & 0 & 0 \\
0 & 0 & 0 & 0 & 0 & 0 & 0 & 0 & 0 & 0 & 0 & 0 &\dashline& 0 & \text{-}1 & 0 & 0 & 0 & 1 & 0 & 0 & 0 & 0 & 0 & 0 \\
0 & 0 & 0 & 0 & 0 & 0 & 0 & 0 & 0 & 0 & 0 & 0 &\dashline& 0 & 0 & \text{-}1 & 0 & 0 & 0 & 1 & 0 & 0 & 0 & 0 & 0 \\
0 & 0 & 0 & 0 & 0 & 0 & 0 & 0 & 0 & 0 & 0 & 0 &\dashline& 0 & 0 & 0 & \text{-}1 & 0 & 0 & 0 & 1 & 0 & 0 & 0 & 0 \\
0 & 0 & 0 & 0 & 0 & 0 & 0 & 0 & 0 & 0 & 0 & 0 &\dashline& 0 & 0 & 0 & 0 & \text{-}1 & 0 & 0 & 0 & 1 & 0 & 0 & 0 \\
0 & 0 & 0 & 0 & 0 & 0 & 0 & 0 & 0 & 0 & 0 & 0 &\dashline& 0 & 0 & 0 & 0 & 0 & \text{-}1 & 0 & 0 & 0 & 1 & 0 & 0 \\
0 & 0 & 0 & 0 & 0 & 0 & 0 & 0 & 0 & 0 & 0 & 0 &\dashline& 0 & 0 & 0 & 0 & 0 & 0 & \text{-}1 & 0 & 0 & 0 & 1 & 0 \\
0 & 0 & 0 & 0 & 0 & 0 & 0 & 0 & 0 & 0 & 0 & 0 &\dashline& 0 & 0 & 0 & 0 & 0 & 0 & 0 & \text{-}1 & 0 & 0 & 0 & 1 \\
0 & 0 & 0 & 0 & 0 & 0 & 0 & 0 & 0 & 0 & 0 & 0 &\dashline& 0 & 0 & 0 & 0 & 0 & 0 & 0 & 0 & \text{-}1 & 0 & 0 & 0 \\
0 & 0 & 0 & 0 & 0 & 0 & 0 & 0 & 0 & 0 & 0 & 0 &\dashline& 0 & 0 & 0 & 0 & 0 & 0 & 0 & 0 & 0 & \text{-}1 & 0 & 0 \\
0 & 0 & 0 & 0 & 0 & 0 & 0 & 0 & 0 & 0 & 0 & 0 &\dashline& 0 & 0 & 0 & 0 & 0 & 0 & 0 & 0 & 0 & 0 & \text{-}1 & 0 \\
0 & 0 & 0 & 0 & 0 & 0 & 0 & 0 & 0 & 0 & 0 & 0 &\dashline& 0 & 0 & 0 & 0 & 0 & 0 & 0 & 0 & 0 & 0 & 0 & \text{-}1 \\
\end{psmallmatrix}
\end{align*}

\begin{align*}
	\underline{D}_{\,y,f}^4 &= \begin{psmallmatrix}
\text{-}\frac{9}{8} & 0 & 0 & 0 & \frac{9}{8} & 0 & 0 & 0 & \text{-}\frac{1}{24} & 0 & 0 & 0 &\dashline& 0 & 0 & 0 & 0 & 0 & 0 & 0 & 0 & 0 & 0 & 0 & 0 \\
0 & \text{-}\frac{9}{8} & 0 & 0 & 0 & \frac{9}{8} & 0 & 0 & 0 & \text{-}\frac{1}{24} & 0 & 0 &\dashline& 0 & 0 & 0 & 0 & 0 & 0 & 0 & 0 & 0 & 0 & 0 & 0 \\
0 & 0 & \text{-}\frac{9}{8} & 0 & 0 & 0 & \frac{9}{8} & 0 & 0 & 0 & \text{-}\frac{1}{24} & 0 &\dashline& 0 & 0 & 0 & 0 & 0 & 0 & 0 & 0 & 0 & 0 & 0 & 0 \\
0 & 0 & 0 & \text{-}\frac{9}{8} & 0 & 0 & 0 & \frac{9}{8} & 0 & 0 & 0 & \text{-}\frac{1}{24} &\dashline& 0 & 0 & 0 & 0 & 0 & 0 & 0 & 0 & 0 & 0 & 0 & 0 \\
\frac{1}{24} & 0 & 0 & 0 & \text{-}\frac{9}{8} & 0 & 0 & 0 & \frac{9}{8} & 0 & 0 & 0 &\dashline& 0 & 0 & 0 & 0 & 0 & 0 & 0 & 0 & 0 & 0 & 0 & 0 \\
0 & \frac{1}{24} & 0 & 0 & 0 & \text{-}\frac{9}{8} & 0 & 0 & 0 & \frac{9}{8} & 0 & 0 &\dashline& 0 & 0 & 0 & 0 & 0 & 0 & 0 & 0 & 0 & 0 & 0 & 0 \\
0 & 0 & \frac{1}{24} & 0 & 0 & 0 & \text{-}\frac{9}{8} & 0 & 0 & 0 & \frac{9}{8} & 0 &\dashline& 0 & 0 & 0 & 0 & 0 & 0 & 0 & 0 & 0 & 0 & 0 & 0 \\
0 & 0 & 0 & \frac{1}{24} & 0 & 0 & 0 & \text{-}\frac{9}{8} & 0 & 0 & 0 & \frac{9}{8} &\dashline& 0 & 0 & 0 & 0 & 0 & 0 & 0 & 0 & 0 & 0 & 0 & 0 \\
0 & 0 & 0 & 0 & \frac{1}{24} & 0 & 0 & 0 & \text{-}\frac{9}{8} & 0 & 0 & 0 &\dashline& 0 & 0 & 0 & 0 & 0 & 0 & 0 & 0 & 0 & 0 & 0 & 0 \\
0 & 0 & 0 & 0 & 0 & \frac{1}{24} & 0 & 0 & 0 & \text{-}\frac{9}{8} & 0 & 0 &\dashline& 0 & 0 & 0 & 0 & 0 & 0 & 0 & 0 & 0 & 0 & 0 & 0 \\
0 & 0 & 0 & 0 & 0 & 0 & \frac{1}{24} & 0 & 0 & 0 & \text{-}\frac{9}{8} & 0 &\dashline& 0 & 0 & 0 & 0 & 0 & 0 & 0 & 0 & 0 & 0 & 0 & 0 \\
0 & 0 & 0 & 0 & 0 & 0 & 0 & \frac{1}{24} & 0 & 0 & 0 & \text{-}\frac{9}{8} &\dashline& 0 & 0 & 0 & 0 & 0 & 0 & 0 & 0 & 0 & 0 & 0 & 0 \\\hline
0 & 0 & 0 & 0 & 0 & 0 & 0 & 0 & 0 & 0 & 0 & 0 &\dashline& \text{-}\frac{9}{8} & 0 & 0 & 0 & \frac{9}{8} & 0 & 0 & 0 & \text{-}\frac{1}{24} & 0 & 0 & 0 \\
0 & 0 & 0 & 0 & 0 & 0 & 0 & 0 & 0 & 0 & 0 & 0 &\dashline& 0 & \text{-}\frac{9}{8} & 0 & 0 & 0 & \frac{9}{8} & 0 & 0 & 0 & \text{-}\frac{1}{24} & 0 & 0 \\
0 & 0 & 0 & 0 & 0 & 0 & 0 & 0 & 0 & 0 & 0 & 0 &\dashline& 0 & 0 & \text{-}\frac{9}{8} & 0 & 0 & 0 & \frac{9}{8} & 0 & 0 & 0 & \text{-}\frac{1}{24} & 0 \\
0 & 0 & 0 & 0 & 0 & 0 & 0 & 0 & 0 & 0 & 0 & 0 &\dashline& 0 & 0 & 0 & \text{-}\frac{9}{8} & 0 & 0 & 0 & \frac{9}{8} & 0 & 0 & 0 & \text{-}\frac{1}{24} \\
0 & 0 & 0 & 0 & 0 & 0 & 0 & 0 & 0 & 0 & 0 & 0 &\dashline& \frac{1}{24} & 0 & 0 & 0 & \text{-}\frac{9}{8} & 0 & 0 & 0 & \frac{9}{8} & 0 & 0 & 0 \\
0 & 0 & 0 & 0 & 0 & 0 & 0 & 0 & 0 & 0 & 0 & 0 &\dashline& 0 & \frac{1}{24} & 0 & 0 & 0 & \text{-}\frac{9}{8} & 0 & 0 & 0 & \frac{9}{8} & 0 & 0 \\
0 & 0 & 0 & 0 & 0 & 0 & 0 & 0 & 0 & 0 & 0 & 0 &\dashline& 0 & 0 & \frac{1}{24} & 0 & 0 & 0 & \text{-}\frac{9}{8} & 0 & 0 & 0 & \frac{9}{8} & 0 \\
0 & 0 & 0 & 0 & 0 & 0 & 0 & 0 & 0 & 0 & 0 & 0 &\dashline& 0 & 0 & 0 & \frac{1}{24} & 0 & 0 & 0 & \text{-}\frac{9}{8} & 0 & 0 & 0 & \frac{9}{8} \\
0 & 0 & 0 & 0 & 0 & 0 & 0 & 0 & 0 & 0 & 0 & 0 &\dashline& 0 & 0 & 0 & 0 & \frac{1}{24} & 0 & 0 & 0 & \text{-}\frac{9}{8} & 0 & 0 & 0 \\
0 & 0 & 0 & 0 & 0 & 0 & 0 & 0 & 0 & 0 & 0 & 0 &\dashline& 0 & 0 & 0 & 0 & 0 & \frac{1}{24} & 0 & 0 & 0 & \text{-}\frac{9}{8} & 0 & 0 \\
0 & 0 & 0 & 0 & 0 & 0 & 0 & 0 & 0 & 0 & 0 & 0 &\dashline& 0 & 0 & 0 & 0 & 0 & 0 & \frac{1}{24} & 0 & 0 & 0 & \text{-}\frac{9}{8} & 0 \\
0 & 0 & 0 & 0 & 0 & 0 & 0 & 0 & 0 & 0 & 0 & 0 &\dashline& 0 & 0 & 0 & 0 & 0 & 0 & 0 & \frac{1}{24} & 0 & 0 & 0 & \text{-}\frac{9}{8} \\
\end{psmallmatrix}
\end{align*}

\begin{align*}
	\underline{D}_{\,z,f}^2 &= \begin{psmallmatrix}
\text{-}1 & 0 & 0 & 0 & 0 & 0 & 0 & 0 & 0 & 0 & 0 & 0 & 1 & 0 & 0 & 0 & 0 & 0 & 0 & 0 & 0 & 0 & 0 & 0 \\
0 & \text{-}1 & 0 & 0 & 0 & 0 & 0 & 0 & 0 & 0 & 0 & 0 & 0 & 1 & 0 & 0 & 0 & 0 & 0 & 0 & 0 & 0 & 0 & 0 \\
0 & 0 & \text{-}1 & 0 & 0 & 0 & 0 & 0 & 0 & 0 & 0 & 0 & 0 & 0 & 1 & 0 & 0 & 0 & 0 & 0 & 0 & 0 & 0 & 0 \\
0 & 0 & 0 & \text{-}1 & 0 & 0 & 0 & 0 & 0 & 0 & 0 & 0 & 0 & 0 & 0 & 1 & 0 & 0 & 0 & 0 & 0 & 0 & 0 & 0 \\
0 & 0 & 0 & 0 & \text{-}1 & 0 & 0 & 0 & 0 & 0 & 0 & 0 & 0 & 0 & 0 & 0 & 1 & 0 & 0 & 0 & 0 & 0 & 0 & 0 \\
0 & 0 & 0 & 0 & 0 & \text{-}1 & 0 & 0 & 0 & 0 & 0 & 0 & 0 & 0 & 0 & 0 & 0 & 1 & 0 & 0 & 0 & 0 & 0 & 0 \\
0 & 0 & 0 & 0 & 0 & 0 & \text{-}1 & 0 & 0 & 0 & 0 & 0 & 0 & 0 & 0 & 0 & 0 & 0 & 1 & 0 & 0 & 0 & 0 & 0 \\
0 & 0 & 0 & 0 & 0 & 0 & 0 & \text{-}1 & 0 & 0 & 0 & 0 & 0 & 0 & 0 & 0 & 0 & 0 & 0 & 1 & 0 & 0 & 0 & 0 \\
0 & 0 & 0 & 0 & 0 & 0 & 0 & 0 & \text{-}1 & 0 & 0 & 0 & 0 & 0 & 0 & 0 & 0 & 0 & 0 & 0 & 1 & 0 & 0 & 0 \\
0 & 0 & 0 & 0 & 0 & 0 & 0 & 0 & 0 & \text{-}1 & 0 & 0 & 0 & 0 & 0 & 0 & 0 & 0 & 0 & 0 & 0 & 1 & 0 & 0 \\
0 & 0 & 0 & 0 & 0 & 0 & 0 & 0 & 0 & 0 & \text{-}1 & 0 & 0 & 0 & 0 & 0 & 0 & 0 & 0 & 0 & 0 & 0 & 1 & 0 \\
0 & 0 & 0 & 0 & 0 & 0 & 0 & 0 & 0 & 0 & 0 & \text{-}1 & 0 & 0 & 0 & 0 & 0 & 0 & 0 & 0 & 0 & 0 & 0 & 1 \\
0 & 0 & 0 & 0 & 0 & 0 & 0 & 0 & 0 & 0 & 0 & 0 & \text{-}1 & 0 & 0 & 0 & 0 & 0 & 0 & 0 & 0 & 0 & 0 & 0 \\
0 & 0 & 0 & 0 & 0 & 0 & 0 & 0 & 0 & 0 & 0 & 0 & 0 & \text{-}1 & 0 & 0 & 0 & 0 & 0 & 0 & 0 & 0 & 0 & 0 \\
0 & 0 & 0 & 0 & 0 & 0 & 0 & 0 & 0 & 0 & 0 & 0 & 0 & 0 & \text{-}1 & 0 & 0 & 0 & 0 & 0 & 0 & 0 & 0 & 0 \\
0 & 0 & 0 & 0 & 0 & 0 & 0 & 0 & 0 & 0 & 0 & 0 & 0 & 0 & 0 & \text{-}1 & 0 & 0 & 0 & 0 & 0 & 0 & 0 & 0 \\
0 & 0 & 0 & 0 & 0 & 0 & 0 & 0 & 0 & 0 & 0 & 0 & 0 & 0 & 0 & 0 & \text{-}1 & 0 & 0 & 0 & 0 & 0 & 0 & 0 \\
0 & 0 & 0 & 0 & 0 & 0 & 0 & 0 & 0 & 0 & 0 & 0 & 0 & 0 & 0 & 0 & 0 & \text{-}1 & 0 & 0 & 0 & 0 & 0 & 0 \\
0 & 0 & 0 & 0 & 0 & 0 & 0 & 0 & 0 & 0 & 0 & 0 & 0 & 0 & 0 & 0 & 0 & 0 & \text{-}1 & 0 & 0 & 0 & 0 & 0 \\
0 & 0 & 0 & 0 & 0 & 0 & 0 & 0 & 0 & 0 & 0 & 0 & 0 & 0 & 0 & 0 & 0 & 0 & 0 & \text{-}1 & 0 & 0 & 0 & 0 \\
0 & 0 & 0 & 0 & 0 & 0 & 0 & 0 & 0 & 0 & 0 & 0 & 0 & 0 & 0 & 0 & 0 & 0 & 0 & 0 & \text{-}1 & 0 & 0 & 0 \\
0 & 0 & 0 & 0 & 0 & 0 & 0 & 0 & 0 & 0 & 0 & 0 & 0 & 0 & 0 & 0 & 0 & 0 & 0 & 0 & 0 & \text{-}1 & 0 & 0 \\
0 & 0 & 0 & 0 & 0 & 0 & 0 & 0 & 0 & 0 & 0 & 0 & 0 & 0 & 0 & 0 & 0 & 0 & 0 & 0 & 0 & 0 & \text{-}1 & 0 \\
0 & 0 & 0 & 0 & 0 & 0 & 0 & 0 & 0 & 0 & 0 & 0 & 0 & 0 & 0 & 0 & 0 & 0 & 0 & 0 & 0 & 0 & 0 & \text{-}1 \\
\end{psmallmatrix}
\end{align*}

\begin{align*}
	\underline{D}_{\,z,f}^4 &= \begin{psmallmatrix}
\text{-}\frac{9}{8} & 0 & 0 & 0 & 0 & 0 & 0 & 0 & 0 & 0 & 0 & 0 & \frac{9}{8} & 0 & 0 & 0 & 0 & 0 & 0 & 0 & 0 & 0 & 0 & 0 \\
0 & \text{-}\frac{9}{8} & 0 & 0 & 0 & 0 & 0 & 0 & 0 & 0 & 0 & 0 & 0 & \frac{9}{8} & 0 & 0 & 0 & 0 & 0 & 0 & 0 & 0 & 0 & 0 \\
0 & 0 & \text{-}\frac{9}{8} & 0 & 0 & 0 & 0 & 0 & 0 & 0 & 0 & 0 & 0 & 0 & \frac{9}{8} & 0 & 0 & 0 & 0 & 0 & 0 & 0 & 0 & 0 \\
0 & 0 & 0 & \text{-}\frac{9}{8} & 0 & 0 & 0 & 0 & 0 & 0 & 0 & 0 & 0 & 0 & 0 & \frac{9}{8} & 0 & 0 & 0 & 0 & 0 & 0 & 0 & 0 \\
0 & 0 & 0 & 0 & \text{-}\frac{9}{8} & 0 & 0 & 0 & 0 & 0 & 0 & 0 & 0 & 0 & 0 & 0 & \frac{9}{8} & 0 & 0 & 0 & 0 & 0 & 0 & 0 \\
0 & 0 & 0 & 0 & 0 & \text{-}\frac{9}{8} & 0 & 0 & 0 & 0 & 0 & 0 & 0 & 0 & 0 & 0 & 0 & \frac{9}{8} & 0 & 0 & 0 & 0 & 0 & 0 \\
0 & 0 & 0 & 0 & 0 & 0 & \text{-}\frac{9}{8} & 0 & 0 & 0 & 0 & 0 & 0 & 0 & 0 & 0 & 0 & 0 & \frac{9}{8} & 0 & 0 & 0 & 0 & 0 \\
0 & 0 & 0 & 0 & 0 & 0 & 0 & \text{-}\frac{9}{8} & 0 & 0 & 0 & 0 & 0 & 0 & 0 & 0 & 0 & 0 & 0 & \frac{9}{8} & 0 & 0 & 0 & 0 \\
0 & 0 & 0 & 0 & 0 & 0 & 0 & 0 & \text{-}\frac{9}{8} & 0 & 0 & 0 & 0 & 0 & 0 & 0 & 0 & 0 & 0 & 0 & \frac{9}{8} & 0 & 0 & 0 \\
0 & 0 & 0 & 0 & 0 & 0 & 0 & 0 & 0 & \text{-}\frac{9}{8} & 0 & 0 & 0 & 0 & 0 & 0 & 0 & 0 & 0 & 0 & 0 & \frac{9}{8} & 0 & 0 \\
0 & 0 & 0 & 0 & 0 & 0 & 0 & 0 & 0 & 0 & \text{-}\frac{9}{8} & 0 & 0 & 0 & 0 & 0 & 0 & 0 & 0 & 0 & 0 & 0 & \frac{9}{8} & 0 \\
0 & 0 & 0 & 0 & 0 & 0 & 0 & 0 & 0 & 0 & 0 & \text{-}\frac{9}{8} & 0 & 0 & 0 & 0 & 0 & 0 & 0 & 0 & 0 & 0 & 0 & \frac{9}{8} \\
\frac{1}{24} & 0 & 0 & 0 & 0 & 0 & 0 & 0 & 0 & 0 & 0 & 0 & \text{-}\frac{9}{8} & 0 & 0 & 0 & 0 & 0 & 0 & 0 & 0 & 0 & 0 & 0 \\
0 & \frac{1}{24} & 0 & 0 & 0 & 0 & 0 & 0 & 0 & 0 & 0 & 0 & 0 & \text{-}\frac{9}{8} & 0 & 0 & 0 & 0 & 0 & 0 & 0 & 0 & 0 & 0 \\
0 & 0 & \frac{1}{24} & 0 & 0 & 0 & 0 & 0 & 0 & 0 & 0 & 0 & 0 & 0 & \text{-}\frac{9}{8} & 0 & 0 & 0 & 0 & 0 & 0 & 0 & 0 & 0 \\
0 & 0 & 0 & \frac{1}{24} & 0 & 0 & 0 & 0 & 0 & 0 & 0 & 0 & 0 & 0 & 0 & \text{-}\frac{9}{8} & 0 & 0 & 0 & 0 & 0 & 0 & 0 & 0 \\
0 & 0 & 0 & 0 & \frac{1}{24} & 0 & 0 & 0 & 0 & 0 & 0 & 0 & 0 & 0 & 0 & 0 & \text{-}\frac{9}{8} & 0 & 0 & 0 & 0 & 0 & 0 & 0 \\
0 & 0 & 0 & 0 & 0 & \frac{1}{24} & 0 & 0 & 0 & 0 & 0 & 0 & 0 & 0 & 0 & 0 & 0 & \text{-}\frac{9}{8} & 0 & 0 & 0 & 0 & 0 & 0 \\
0 & 0 & 0 & 0 & 0 & 0 & \frac{1}{24} & 0 & 0 & 0 & 0 & 0 & 0 & 0 & 0 & 0 & 0 & 0 & \text{-}\frac{9}{8} & 0 & 0 & 0 & 0 & 0 \\
0 & 0 & 0 & 0 & 0 & 0 & 0 & \frac{1}{24} & 0 & 0 & 0 & 0 & 0 & 0 & 0 & 0 & 0 & 0 & 0 & \text{-}\frac{9}{8} & 0 & 0 & 0 & 0 \\
0 & 0 & 0 & 0 & 0 & 0 & 0 & 0 & \frac{1}{24} & 0 & 0 & 0 & 0 & 0 & 0 & 0 & 0 & 0 & 0 & 0 & \text{-}\frac{9}{8} & 0 & 0 & 0 \\
0 & 0 & 0 & 0 & 0 & 0 & 0 & 0 & 0 & \frac{1}{24} & 0 & 0 & 0 & 0 & 0 & 0 & 0 & 0 & 0 & 0 & 0 & \text{-}\frac{9}{8} & 0 & 0 \\
0 & 0 & 0 & 0 & 0 & 0 & 0 & 0 & 0 & 0 & \frac{1}{24} & 0 & 0 & 0 & 0 & 0 & 0 & 0 & 0 & 0 & 0 & 0 & \text{-}\frac{9}{8} & 0 \\
0 & 0 & 0 & 0 & 0 & 0 & 0 & 0 & 0 & 0 & 0 & \frac{1}{24} & 0 & 0 & 0 & 0 & 0 & 0 & 0 & 0 & 0 & 0 & 0 & \text{-}\frac{9}{8} \\
\end{psmallmatrix}
\end{align*}

Außerhalb der senkrechten gestrichelten Linien sind alle Einträge Null, unabhängig von der Ordnung der FD-Operatoren. Der Grund hierfür ist der Tatsache geschuldet, dass an diesen Stellen ein Sprung $y\rightarrow y+1$ nach $NX$ Einträgen bei $\underline{D}_{\,x}$ oder $z\rightarrow z+1$ nach $NX \cdot NY$ Einträgen bei $\underline{D}_{\,y}$ stattfindet.

Alle $\underline{D}_{\,i,b}$ lassen sich aus $\underline{D}_{\,i,f}$ ableiten, indem alle Einträge ungleich Null einer Zeile durch ihren rechten Nachfolger ungleich Null ersetzt werden. Dass dies einer Multiplikation mit $-1$ und einem Transponieren entspricht wird besonders bei $\underline{D}_{\,y}^4$ deutlich.


Eine wichtige Eigenschaft der FD-Koeffizienten-Matrizen liegt in ihrer Behandlung des Modellrandes über welchen sie niemals hinaus greifen. Dies führt zwangsläufig dazu, dass das Update von $v_i$ und $\sigma_{ij}$ in der Nähe des Randes aus einem unvollständigen FD-Operator berechnet wird. So wird für $v_x^{n+1}$ in der letzten $y-z-$Ebene mit $k=NX$ nicht vorgegriffen und bei einem FD-Operator 2. Ordnung lediglich je ein Wert von $\sigma_{xx}^n$, $\sigma_{xy}^n$ und $\sigma_{xz}^n$ verwendet. Das nicht Beachten aller Größen außerhalb des Randes entspricht einem Festhalten des Wellenfeldes und damit der festen Einspannung des Mediums. Die Folge ist eine vollständige Reflexion.

Alle diskretisierte Komponenten der elastischen Wellengleichung in Matrix-Vektor Schreibweise lauten:
\begin{align*}
	\vec{\sigma}_{xx}^{\,n} &= \vec{\sigma}_{xx}^{\,n-1} + \frac{\Delta t}{\Delta h}~ \mathrm{diag} \left( \vec{\lambda}^{\,T} \right) \cdot \left( \underline{D}_{\,x,b}^q \vec{v}_x +\underline{D}_{\,y,b}^q \vec{v}_y + \underline{D}_{\,z,b}^q \vec{v}_z \right) + 2~ \frac{\Delta t}{\Delta h} ~\mathrm{diag} \left( \vec{\mu}^{\,T} \right) \cdot \underline{D}_{\,x,b}^q \vec{v}_x\\
	\vec{\sigma}_{yy}^{\,n} &= \vec{\sigma}_{yy}^{\,n-1} + \frac{\Delta t}{\Delta h}~ \mathrm{diag} \left( \vec{\lambda}^{\,T} \right) \cdot \left( \underline{D}_{\,x,b}^q \vec{v}_x +\underline{D}_{\,y,b}^q \vec{v}_y + \underline{D}_{\,z,b}^q \vec{v}_z \right) + 2~ \frac{\Delta t}{\Delta h} ~\mathrm{diag} \left( \vec{\mu}^{\,T} \right) \cdot \underline{D}_{\,y,b}^q \vec{v}_y\\
	\vec{\sigma}_{zz}^{\,n} &= \vec{\sigma}_{zz}^{\,n-1} + \frac{\Delta t}{\Delta h}~ \mathrm{diag} \left( \vec{\lambda}^{\,T} \right) \cdot \left( \underline{D}_{\,x,b}^q \vec{v}_x +\underline{D}_{\,y,b}^q \vec{v}_y + \underline{D}_{\,z,b}^q \vec{v}_z \right) + 2~ \frac{\Delta t}{\Delta h} ~\mathrm{diag} \left( \vec{\mu}^{\,T} \right) \cdot \underline{D}_{\,z,b}^q \vec{v}_z\\
	\vec{\sigma}_{xy}^{\,n} &= \vec{\sigma}_{xy}^{\,n-1} + \frac{\Delta t}{\Delta h}~ \mathrm{diag} \left( \vec{\mu}^{\,T} \right) \cdot \left( \underline{D}_{\,y,f}^q \vec{v}_x + \underline{D}_{\,x,f}^q \vec{v}_y \right)\\
	\vec{\sigma}_{xz}^{\,n} &= \vec{\sigma}_{xz}^{\,n-1} + \frac{\Delta t}{\Delta h}~ \mathrm{diag} \left( \vec{\mu}^{\,T} \right) \cdot \left( \underline{D}_{\,z,f}^q \vec{v}_x + \underline{D}_{\,x,f}^q \vec{v}_z \right)\\
	\vec{\sigma}_{yz}^{\,n} &= \vec{\sigma}_{yz}^{\,n-1} + \frac{\Delta t}{\Delta h}~ \mathrm{diag} \left( \vec{\mu}^{\,T} \right) \cdot \left( \underline{D}_{\,z,f}^q \vec{v}_y + \underline{D}_{\,y,f}^q \vec{v}_z \right)\\
	\vec{v}_x^{\,n+1} &= \vec{v}_x^{\,n} + \frac{\Delta t}{\Delta h} ~ \mathrm{diag} \left( \vec{\rho}_\mathrm{inv}^{\,T} \right) \cdot \left( \underline{D}_{\,x,f}^q \vec{\sigma}_{xx}^{\,n} + \underline{D}_{\,y,b}^q \vec{\sigma}_{xy}^{\,n} + \underline{D}_{\,z,b}^q \vec{\sigma}_{xz}^{\,n} \right)\\
	\vec{v}_y^{\,n+1} &= \vec{v}_y^{\,n} + \frac{\Delta t}{\Delta h} ~ \mathrm{diag} \left( \vec{\rho}_\mathrm{inv}^{\,T} \right) \cdot \left( \underline{D}_{\,x,b}^q \vec{\sigma}_{yx}^{\,n} + \underline{D}_{\,y,f}^q \vec{\sigma}_{yy}^{\,n} + \underline{D}_{\,z,b}^q \vec{\sigma}_{yz}^{\,n} \right)\\
	\vec{v}_z^{\,n+1} &= \vec{v}_z^{\,n} + \frac{\Delta t}{\Delta h} ~ \mathrm{diag} \left( \vec{\rho}_\mathrm{inv}^{\,T} \right) \cdot \left( \underline{D}_{\,x,b}^q \vec{\sigma}_{zx}^{\,n} + \underline{D}_{\,y,b}^q \vec{\sigma}_{zy}^{\,n} + \underline{D}_{\,z,f}^q \vec{\sigma}_{zz}^{\,n} \right)\\
\end{align*}

\newpage

Acoustic version:
\begin{align*}
	\vec{p}^{\,n} &= \vec{p}^{\,n-1} + \frac{\Delta t}{\Delta h}~ \mathrm{diag} \left( \vec{\lambda}^{\,T} \right) \cdot \left( \underline{D}_{\,x,b}^q \vec{v}_x +\underline{D}_{\,y,b}^q \vec{v}_y + \underline{D}_{\,z,b}^q \vec{v}_z \right) \\
	\vec{v}_x^{\,n+1} &= \vec{v}_x^{\,n} + \frac{\Delta t}{\Delta h} ~ \mathrm{diag} \left( \vec{\rho}_\mathrm{inv}^{\,T} \right) \cdot \left( \underline{D}_{\,x,f}^q \vec{p}^{\,n}  \right)\\
	\vec{v}_y^{\,n+1} &= \vec{v}_y^{\,n} + \frac{\Delta t}{\Delta h} ~ \mathrm{diag} \left( \vec{\rho}_\mathrm{inv}^{\,T} \right) \cdot \left( \underline{D}_{\,y,f}^q \vec{p}^{\,n} \right)\\
	\vec{v}_z^{\,n+1} &= \vec{v}_z^{\,n} + \frac{\Delta t}{\Delta h} ~ \mathrm{diag}  \left( \vec{\rho}_\mathrm{inv}^{\,T} \right) \cdot\left( \underline{D}_{\,z,f}^q \vec{p}^{\,n} \right)\\
\end{align*}

\cleardoublepage
\bibliography{WAVE_theory}
\bibliographystyle{apalike}
\end{document}